\section{Planteo del problema}
\normalsize{ \indent
En la actualidad, la Universidad Nacional de Misiones
gestiona los papeles de los servicios a trav\'es
de documentos generados en un programa ofim\'atico,
sin haber otro tipo de informatizaci\'on sobre
el proceso. Al no tener un sistema implantado para
este prop\'osito, se carece de control sobre las
distintas etapas en que la realizaci\'on de un
servicio se encuentra.
}
\newline
\normalsize{ \indent
Adicionalmente, el tiempo que transcurre ente cada
uno de los pasos a seguir en el procedimiento de un
servicio es amplio; considerando las formas en que
las solicitudes de servicio llegan, sea por los
docentes que obtienen las solicitudes de servicios
o por el acercamiento presencial del solicitante.
}
\section{Introducci\'on y Objetivos}
\normalsize{ \indent
La Universidad Nacional de Misiones realiza servicios
a terceros que est\'en interesados en concretar uno.
Los mismos pueden clasificarse en aquellos
generados por \'ordenes y por convenio.
Las \'ordenes de servicio son documentos que
constatan la vinculaci\'on entre Comitentes y
Responsables T\'ecnicos, agregando que la persona
a cargo de la Secretar\'ia de Extensi\'on y
Vinculaci\'on Tecnol\'ogica est\'a en conocimiento
de esto. Por otro lado, los convenios provienen
de resoluciones solventadas por el Consejo Directivo
para acordar la realizaci\'on de servicios.
}
\newline
\normalsize{ \indent
En la b\'usqueda de formalizar el circuito que
involucra al servicio, se realizar\'a un sistema
inform\'atico con los siguientes objetivos:
}
\begin{itemize}
	\item Disponer de una plantilla de orden
	de servicio parametrizada, para generar
	el documento de forma consistente.
	\item Formalizar una v\'ia r\'apida para
	acordar t\'erminos de servicios.
	\item Validar las firmas en los documentos
	para \'ordenes de servicios.
	\item Definir condiciones en donde se
	suspenda una orden de servicio o su
	solicitud.
	\item Poseer los servicios realizados
	por orden y por convenios en un solo
	lugar.
	\item Vincular cada uno de los
	comprobantes con la etapa en la que se
	encuentra un servicio.
\end{itemize}
\section{Alcance}
\normalsize{ \indent
Para el producto software a desarrollar, se va
a considerar:
}
\begin{itemize}
	\item Seguimiento de los servicios por
	convenio.
	\item Generaci\'on de la orden de servicio.
	\item Registro de los costos involucrados en
	el servicio.
	\item Negociaci\'on de los recursos,
	compromisos y retribuciones de un servicio.
	\item Firma de las \'ordenes de servicio.
	\item Anexo de factura y recibo
	correspondiente por orden de servicio.
\end{itemize}
\section{Limitaciones}
\normalsize{ \indent
Las cuestiones que no ser\'an abarcadas en, al
menos, una versi\'on inicial son:
}
\begin{itemize}
	\item Revisi\'on de ART de los trabajadores
	de planta o con relaci\'on de dependencia,
	y seguros de monotributistas o becarios.
\end{itemize}
\ \newline
\normalsize{ \indent
Las limitaciones definitivas del sistema (es
decir, que nunca ser\'an parte de este software)
son:
}
\begin{itemize}
	\item La facturaci\'on de la orden de
	servicio, que ya se realiza por un sistema
	de AFIP.
	\item Medios de cobro a trav\'es del sistema
	(tales como MercadoPago u otros medios de
	transferencia) para las \'ordenes de servicio.
	\item Medios de pago para retribuir a los
	miembros que prestaron su servicio.
\end{itemize}
\section{Especificaci\'on de M\'odulos}
\subsection[Funcionales]{Especificaciones Funcionales}
\begin{itemize}
	\item \'Ordenes de servicio
	\item Convenios
	\item Presupuestos
	\item Cobros
	\item Formularios
	\item Avisos peri\'odicos
	\item Notificaciones
\end{itemize}
\subsection[No Funcionales]{Especificaciones No Funcionales}
\begin{itemize}
	\item Estad\'istica
	\item Ajuste de par\'ametros
	\item Auditor\'ia
	\item Seguridad
	\item Gesti\'on de roles:
	\begin{itemize}
		\item Administrador
		\item Representante T\'ecnico
		\item Comitente
		\item Secretario
		\item Ayudante
	\end{itemize}
\end{itemize}
\section{Descripci\'on de los M\'odulos}
\begin{itemize}
	\item \textbf{\'Ordenes de servicio:}
	almacena informaci\'on sobre las \'ordenes
	de servicio, incluyendo las firmas del mismo.
	Su salida genera informes sobre el estado de
	cada orden de servicio.
	\item \textbf{Convenios:} resguarda los datos
	sobre los servicios que se proceden por
	resoluci\'on.
	\item \textbf{Presupuestos:} guarda los
	montos de cada recurso involucrado para un
	servicio, justificando el precio de la
	retribuci\'on econ\'omica.
	\item \textbf{Cobros:} mantiene un seguimiento
	de las retribuciones econ\'omicas de los
	comitentes, desde el monto inicial establecido
	hasta cubrir su totalidad.
	\item \textbf{Formularios:} ofrece interfaces
	para poder completar los datos que pertenecen
	a la orden de servicio, as\'i puede generarse
	tal documento con los datos completados.
	\item \textbf{Avisos peri\'odicos:} env\'ia
	correos electr\'onicos a los usuarios sobre
	distintos estados en que se encuentra el
	sistema.
	\item \textbf{Notificaciones:} conserva los
	m\'etodos utilizados para avisar a los
	distintos usuarios del sistema sobre las
	situaciones que pueden ocurrir, dentro del
	ecosistema del software.
	\item \textbf{Estad\'istica:} se dedica a los
	c\'alculos estad\'isticos sobre la completitud
	de servicios, montos de servicios y servicios
	por \'area laboral. Su salida genera los
	informes estad\'isticos solicitados.
	\item \textbf{Ajuste de par\'ametros:} guarda
	cada uno de los par\'ametros ajustables que
	involucran otros m\'odulos. Como salida, se
	devuelve el conjunto de variables globales y
	parametrizables dentro del software.
	\item \textbf{Auditor\'ia:} registra cada una
	de las acciones que se realizan dentro del
	sistema, para controlar los sucesos que vayan
	ocurriendo a lo largo del uso del mismo.
	\item \textbf{Seguridad:} asegura que el
	ingreso de cualquier usuario que intente
	acceder al sistema sea v\'alido, as\'i como
	limitar el contenido disponible para cada rol.
\end{itemize}
\newpage
\subsection*{Gesti\'on de roles}
\begin{itemize}
	\item \textbf{Administrador:} parametriza los
	valores necesarios para que el sistema pueda
	funcionar con flexibilidad, pero sin perder
	las caracter\'isticas deseadas. Se trata de
	las operaciones ABM y CRUD, de los
	par\'ametros ajustables, etc\'etera.
	\item \textbf{Representante T\'ecnico:} tiene
	el poder de adjudicaci\'on a los servicios
	que la universidad realice, sea elegido por
	el Comitente o se apunte al servicio en
	particular; asigna los recursos del servicio,
	y propone los Compromisos y Retribuciones
	de cada servicio que se halle a cargo. Tambi\'en
	es una de las partes que firma una orden de
	servicio.
	\item \textbf{Comitente:} solicita \'ordenes
	de servicio a la universidad, pudiendo elegir
	Representantes T\'ecnicos (en caso de desear
	uno en particular) y otros Comitentes para ser
	involucrados en el servicio; as\'i como decide qu\'e
	acci\'on tomar ante la propuesta de Compromisos
	y Retribuciones de un servicio, sea renegociar,
	aceptar o rechazar la propuesta. Tambi\'en es
	una de las partes que firma una orden de
	servicio.
	\item \textbf{Secretario:} revisa cada una de
	las \'ordenes de servicio y otorga su firma
	final en el documento, pudiendo subirse en
	el sistema. Tambi\'en se encarga de tomar
	decisiones en casos de conflicto por las
	\'ordenes de servicios.
	\item \textbf{Ayudante:} se encarga de subir
	las \'ordenes de servicio que hayan sido
	firmadas manuscritamente, las facturas y los
	recibos que corresponden por servicio.
\end{itemize}
\newpage
\section{Procesos automatizados}
\begin{center}
\begin{longtable}{
	| p{3.25cm} | p{5.25cm} | p{7.4cm} |
}
	\hline
	\rowcolor{gray}
	\hfil \textbf{Denominaci\'on} &
	\hfil \textbf{M\'odulos que intervienen}
	&
	\hfil \textbf{Descripci\'on del proceso}
	\\
	\hline
	\endhead
	\raggedleft Firma digital &
	\multirowcell{1}{
		\Centerstack{
			\'Ordenes de Servicio \\
			Formularios \\
			Avisos peri\'odicos \\
			Notificaciones \\
			Gesti\'on de roles
		}
	} &
	El sistema administrar\'a la firma
	de \'ordenes de servicios de forma
	digital, cuando los firmantes marquen
	que puede firmar documentos
	digitalmente. Este m\'odulo se
	ejecutar\'a cuando se acepte las
	condiciones propuestas por los
	Representantes T\'ecnicos. Se espera
	que firmen los Comitentes primero,
	despu\'es los Responsables T\'ecnicos
	y por \'ultimo la persona encargada
	de la Secretar\'ia de Extensi\'on.
	\\
	\hline
	\raggedleft
	Verificaci\'on autom\'atica de
	\'ordenes de servicio firmadas
	de forma manuscrita
	&
	\multirowcell{1}{
		\Centerstack{
			\'Ordenes de Servicio \\
			Formularios \\
			Avisos peri\'odicos \\
			Notificaciones \\
			Gesti\'on de roles
		}
	} &
	El sistema controlar\'a el documento
	que se suba como orden de servicio,
	cuando el Comitente marque que no
	puede firmar documentos digitalmente.
	Este m\'odulo se ejecutar\'a cuando
	se firme la orden de servicio en papel.
	Se espera que verifique a qu\'e orden
	de servicio pertenece el documento
	subido, la coincidencia de su contenido
	con el formulario rellenado anteriormente
	y la existencia de las firmas en el papel.
	\\
	\hline
	\raggedleft Gestor autom\'atico de servicios
	&
	\multirowcell{1}{
		\Centerstack{
			Seguros \\
			Formularios \\
			Avisos peri\'odicos \\
			Notificaciones \\
			Gesti\'on de roles
		}
	} &
	El sistema controlar\'a el flujo
	de operaci\'on de un servicio,
	realizando acciones autom\'aticamente
	al tener las condiciones adecuadas.
	Este m\'odulo abarcar\'a desde la
	solicitud de servicio hasta la firma de
	su orden. Se espera que anuncie las acciones
	a realizar hacia el actor que corresponda,
	suspenda una solicitud u orden de servicio tras
	no cumplir el flujo configurado, permita
	redefinir una proposici\'on de
	Compromisos y Retribuciones al paso de
	cierto tiempo y proporcione renegociaci\'on
	a la solicitud de servicio.
	\\
	\hline
	\newpage
	\raggedleft Sugeridor de servicio por convenio
	&
	\multirowcell{1}{
		\Centerstack{
			\'Ordenes de Servicio \\
			Formularios \\
			Avisos peri\'odicos \\
			Notificaciones \\
			Gesti\'on de roles
		}
	} &
	El sistema controlar\'a la frecuencia con
	la que un Comitente haga un tipo de
	servicio por orden, sugiriendo realizarlo
	por convenios en lugar de por orden.
	Este m\'odulo abarcar\'a servicios que
	lleguen a ser completados. Se espera que
	notifique al Comitente sobre la acci\'on
	repetitiva y recomiende la mejor\'ia.
	\\
	\hline
\end{longtable}
\end{center}
\clearpage
\section{Estimaci\'on de tama\~no por m\'odulo}
\begin{center}
\begin{longtable}{
	| p{7.95cm} | p{7.95cm} |
}
	\hline
	\rowcolor{gray}
	\hfil \textbf{M\'odulo} &
	\hfil
	\textbf{
		Porcentaje de Participaci\'on /
		Producto
	}
	\\
	\hline
	\endhead
	\rowcolor{lightgray}
	\textbf{Total} & 100\% \\
	\hline
	\endfoot
	Firma digital & 20\% \\
	\hline
	Verificaci\'on autom\'atica de \'ordenes de
	servicio firmadas de forma manuscrita & 20\% \\
	\hline
	Gestor autom\'atico de servicios & 10\% \\
	\hline
	Sugeridor de servicio por convenio & 8\% \\
	\hline
	\'Ordenes de servicio & 8\% \\
	\hline
	Convenios & 6\% \\
	\hline
	Presupuestos & 4\% \\
	\hline
	Cobros & 4\% \\
	\hline
	Formularios & 4\% \\
	\hline
	Avisos peri\'odicos & 4\% \\
	\hline
	Notificaciones & 4\% \\
	\hline
	Estad\'istica & 4\% \\
	\hline
	Gesti\'on de roles & 2\% \\
	\hline
	Ajuste de par\'ametros & 2\% \\
	\hline
\end{longtable}
\end{center}
\section{%
	Entorno tecnol\'ogico y metodol\'ogico%
}
\begin{center}
\begin{tabular}{ | e | p{10.3cm} | }
	\hline
	\textbf{Lenguajes de programaci\'on} &
	Python - JavaScript \\
	\hline
	\textbf{Framework} & Django \\
	\hline
	\textbf{Arquitectura} &
	Modelo - Plantilla - Vista \\
	\hline
	\textbf{Motores de Base de Datos} &
	PostgreSQL - Redis \\
	\hline
	\textbf{Metodolog\'ia seleccionada} &
	Proceso Unificado (UP) \\
	\hline
	\textbf{Tipo de proyecto} &
	Con cliente final \\
	\hline
\end{tabular}
\end{center}
\clearpage
\section{Planificaci\'on de actividades}
\begin{center}
\begin{ganttchart}[
	expand chart=\textwidth,
	title/.append style={
		draw=black,
		fill=lightgray
	},
	group/.append style={
		draw=black
	},
	hgrid,
	vgrid={*{6}{draw=none},dotted},
	y unit title=0.5cm,
	y unit chart=0.5cm,
	time slot format=little-endian,
	calendar week text= {\tiny
		{\currentweek}
	},
	title height=1,
	progress label text={},
	bar height=0.4
]{22.04.2024}{20.12.2024}
	\gantttitle[
		title label node/.append style=
			{left=0.1cm and 0.05cm}
	]{A\~no}{0}
	\gantttitlecalendar{year} \\
	\gantttitle[
		title label node/.append style=
			{left=0.1cm and 0.05cm}
	]{Mes}{0}
	\gantttitlecalendar{month} \\
	\gantttitle[
		title label node/.append style=
			{left=0.1cm and 0.05cm}
	]{Semana}{0}
	\gantttitlecalendar{week=17} \\
	\ganttgroup[progress=0]
			{Requerimientos}
			{06.05.2024}{14.06.2024} \\
	\ganttgroup[progress=0]
			{An\'alisis}
			{27.05.2024}{28.06.2024} \\
	\ganttgroup[progress=0]
			{Dise\~no}
			{27.05.2024}{26.07.2024} \\
	\ganttgroup[progress=0]
			{Implementaci\'on}
			{10.06.2024}{27.09.2024} \\
	\ganttgroup[progress=0]
			{Pruebas}
			{24.06.2024}{11.10.2024} \\
	\ganttgroup[progress=0]
			{Implantaci\'on}
			{30.09.2024}{26.10.2024} \\
	\ganttgroup[progress=0]
			{Capacitaci\'on}
			{14.10.2024}{08.11.2024} \\
	\ganttgroup[progress=0]
			{Manuales}
			{24.06.2024}{08.11.2024} \\
	\ganttbar[progress=0]
			{Planificaci\'on}
			{22.04.2024}{03.05.2024} \\
	\ganttbar[progress=0]
			{Factibilidad}
			{06.05.2024}{07.06.2024} \\
	\ganttbar[progress=0]
			{Relevamiento}
			{13.05.2024}{24.05.2024} \\
	\ganttbar[progress=0]
			{Requisitos}
			{20.05.2024}{07.06.2024} \\
	\ganttbar[progress=0]
			{Casos de uso}
			{20.05.2024}{14.06.2024} \\
	\ganttbar[progress=0]
			{Modelo de dominio}
			{27.05.2024}{14.06.2024} \\
	\ganttbar[progress=0]
			{Secuencias de sistema}
			{03.06.2024}{21.06.2024} \\
	\ganttbar[progress=0]
			{Contratos}
			{10.06.2024}{28.06.2024} \\
	\ganttbar[progress=0]
			{Casos de uso reales}
			{27.05.2024}{14.06.2024} \\
	\ganttbar[progress=0]
			{Secuencias de dise\~no}
			{03.06.2024}{21.06.2024} \\
	\ganttbar[progress=0]
			{Diagrama de clases}
			{10.06.2024}{28.06.2024} \\
	\ganttbar[progress=0]
			{Dise\~no de entradas}
			{17.06.2024}{05.07.2024} \\
	\ganttbar[progress=0]
			{Dise\~no de salidas}
			{24.06.2024}{12.07.2024} \\
	\ganttbar[progress=0]
			{Dise\~no de interfaces}
			{01.07.2024}{19.07.2024} \\
	\ganttbar[progress=0]
			{Dise\~no de base de datos}
			{08.07.2024}{26.07.2024} \\
	\ganttbar[progress=0]
			{50\% de implementaci\'on}
			{10.06.2024}{19.07.2024} \\
	\ganttbar[progress=0]
			{75\% de implementaci\'on}
			{22.07.2024}{23.08.2024} \\
	\ganttbar[progress=0]
			{100\% de implementaci\'on}
			{26.08.2024}{27.09.2024} \\
	\ganttbar[progress=0]
			{50\% de pruebas}
			{24.06.2024}{05.08.2024} \\
	\ganttbar[progress=0]
			{75\% de pruebas}
			{08.08.2024}{06.09.2024} \\
	\ganttbar[progress=0]
			{100\% de pruebas}
			{09.09.2024}{11.10.2024} \\
	\ganttbar[progress=0]
			{Implantaci\'on}
			{30.09.2024}{26.10.2024} \\
	\ganttbar[progress=0]
			{Capacitaci\'on}
			{14.10.2024}{08.11.2024} \\
	\ganttbar[progress=0]
			{Manuales}
			{24.06.2024}{08.11.2024} \\
	\ganttbar[progress=0]
			{Ambientaci\'on personal}
			{11.11.2024}{22.11.2024} \\
	\ganttbar[progress=0]
			{Pr\'actica de exposici\'on}
			{25.11.2024}{07.12.2024}
\end{ganttchart}
\end{center}
\clearpage
\begin{center}
\begin{longtable}{
	| p{5.3cm} | p{5.3cm} | p{5.3cm} |
}
	\hline
	\rowcolor{lightgray}
	\hfil \textbf{Actividad} &
	\hfil \textbf{Fecha de inicio} &
	\hfil \textbf{Fecha de finalizaci\'on}
	\\
	\hline
	\endhead
	Planificaci\'on &
	22/04/2024 & 03/05/2024 \\
	\hline
	Factibilidad &
	06/05/2024 & 07/06/2024 \\
	\hline
	Relevamiento &
	13/05/2024 & 24/05/2024 \\
	\hline
	Requisitos &
	20/05/2024 & 07/06/2024 \\
	\hline
	Casos de uso &
	20/05/2024 & 14/06/2024 \\
	\hline
	Modelo de dominio &
	27/05/2024 & 14/06/2024 \\
	\hline
	Secuencias de sistema &
	03/06/2024 & 21/06/2024 \\
	\hline
	Contratos &
	10/06/2024 & 28/06/2024 \\
	\hline
	Casos de uso reales &
	27/05/2024 & 14/06/2024 \\
	\hline
	Secuencias de dise\~no &
	03/06/2024 & 21/06/2024 \\
	\hline
	Diagrama de clases &
	10/06/2024 & 28/06/2024 \\
	\hline
	Dise\~no de entradas &
	17/06/2024 & 05/07/2024 \\
	\hline
	Dise\~no de salidas &
	24/06/2024 & 12/07/2024 \\
	\hline
	Dise\~no de interfaces &
	01/07/2024 & 19/07/2024 \\
	\hline
	Dise\~no de base de datos &
	08/07/2024 & 26/07/2024 \\
	\hline
	50\% de implementaci\'on &
	10/06/2024 & 19/07/2024 \\
	\hline
	75\% de implementaci\'on &
	22/07/2024 & 23/08/2024 \\
	\hline
	100\% de implementaci\'on &
	26/08/2024 & 27/09/2024 \\
	\hline
	50\% de pruebas &
	24/06/2024 & 05/08/2024 \\
	\hline
	75\% de pruebas &
	08/08/2024 & 06/09/2024 \\
	\hline
	100\% de pruebas &
	09/09/2024 & 11/10/2024 \\
	\hline
	Implantaci\'on &
	30/09/2024 & 26/10/2024 \\
	\hline
	Capacitaci\'on &
	14/10/2024 & 08/11/2024 \\
	\hline
	Manuales &
	24/06/2024 & 08/11/2024 \\
	\hline
	Ambientaci\'on personal &
	11/11/2024 & 22/11/2024 \\
	\hline
	Pr\'actica de exposici\'on &
	25/11/2024 & 07/12/2024 \\
	\hline
\end{longtable}
\end{center}
