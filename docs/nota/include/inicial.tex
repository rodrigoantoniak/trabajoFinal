\addcontentsline{toc}{section}{Planteo
	del problema
}
\section*{Planteo del problema}
\normalsize{ \indent
En la actualidad, la Universidad Nacional de Misiones
gestiona las \'ordenes de servicio a trav\'es de
documentos generados en un programa ofim\'atico;
sin haber otro tipo de informatizaci\'on sobre
el proceso. Al no tener un sistema implantado para
este prop\'osito, se carece de control sobre las
distintas etapas en que la realizaci\'on de un
servicio se encuentra.
}
\newline
\normalsize{ \indent
Adicionalmente, el tiempo que transcurre ente cada
uno de los pasos a seguir en el procedimiento de un
servicio es amplio; considerando las formas en que
las solicitudes de servicio llegan, sea por los
docentes que obtienen las solicitudes de servicios
o por el acercamiento presencial del solicitante.
}
\addcontentsline{toc}{section}{Introducci\'on
	y Objetivos
}
\section*{Introducci\'on y Objetivos}
\normalsize{ \indent
Las \'ordenes de servicio son solicitudes que
hacen distintas organizaciones a la Universidad
Nacional de Misiones, buscando resolver sus
necesidades a trav\'es de los recursos
universitarios. En la b\'usqueda de formalizar
el circuito que involucra al servicio, se
realizar\'a un sistema inform\'atico con los
siguientes objetivos:
}
\begin{itemize}
	\item Disponer de una plantilla de orden
	de servicio parametrizada, para generar
	el documento de forma consistente.
	\item Otorgar un v\'ia m\'as formal y
	r\'apida para acordar los t\'erminos
	en que se regir\'a un servicio.
	\item Controlar la validez de las firmas
	en los documentos firmados para las
	\'ordenes de servicio.
	\item Definir condiciones en donde se
	cierre una orden de servicio o su
	solicitud.
	\item Vincular cada uno de los
	comprobantes con la etapa en la que se
	encuentra un servicio.
\end{itemize}
\addcontentsline{toc}{section}{Alcance y
	Limitaciones
}
\section*{Alcance y Limitaciones}
\normalsize{ \indent
Para el producto software a desarrollar, se va
a considerar:
}
\begin{itemize}
	\item Generaci\'on de la orden de servicio.
	\item Registro de los costos invlucrados en
	el servicio.
	\item Negociaci\'on de los recursos,
	compromisos y retribuciones de un servicio.
	\item Firma de las \'ordenes de servicio.
	\item Anexo de factura y recibo
	correspondiente por orden de servicio.
\end{itemize}
\ \newline
\normalsize{ \indent
Las cuestiones que no ser\'an abarcadas en, al
menos, una versi\'on inicial son:
}
\begin{itemize}
	\item La solicitud de los pagos a las
	distintas partes de los Prestadores de
	Servicio.
	\item Revisi\'on de ART de los trabajadores
	de planta o con relaci\'on de dependencia,
	y seguros de monotributistas o becarios.
\end{itemize}
\ \newline
\normalsize{ \indent
Las limitaciones definitivas del sistema (es
decir, que nunca ser\'an parte de este software)
son:
}
\begin{itemize}
	\item La facturaci\'on de la orden de
	servicio, que ya se realiza por un sistema
	de AFIP.
	\item Medios de cobro a trav\'es del sistema
	(tales como MercadoPago u otros medios de
	transferencia) para las \'ordenes de servicio.
	\item Medios de pago para retribuir a los
	miembros que prestaron su servicio.
\end{itemize}
\addcontentsline{toc}{section}{Especificaci\'on
	de M\'odulos
}
\section*{Especificaci\'on de M\'odulos}
\addcontentsline{toc}{subsection}{Funcionales}
\subsection*{Especificaciones Funcionales}
\begin{itemize}
	\item \'Ordenes de servicio
	\item Presupuestos
	\item Cobros
	\item Formularios
	\item Avisos peri\'odicos
	\item Notificaciones
	\item Estad\'istica
	\item Ajuste de par\'ametros
\end{itemize}
\addcontentsline{toc}{subsection}{No Funcionales}
\subsection*{Especificaciones No Funcionales}
\begin{itemize}
	\item Auditor\'ia
	\item Seguridad
	\item Gesti\'on de roles:
	\begin{itemize}
		\item Administrador
		\item Representante T\'ecnico
		\item Comitente
		\item Secretario
		\item Ayudante
		\item Tesorero
	\end{itemize}
\end{itemize}
\addcontentsline{toc}{section}{Descripci\'on
	de los M\'odulos
}
\section*{Descripci\'on de los M\'odulos}
\begin{itemize}
	\item \textbf{\'Ordenes de servicio:}
	almacena informaci\'on sobre las \'ordenes
	de servicio, incluyendo las firmas del mismo.
	Su salida genera informes sobre el estado de
	cada orden de servicio.
	\item \textbf{Presupuestos:} guarda los
	montos de cada recurso involucrado para un
	servicio, justificando el precio de la
	retribuci\'on econ\'omica.
	\item \textbf{Cobros:} mantiene un seguimiento
	de las retribuciones econ\'omicas de los
	comitentes, desde el monto inicial establecido
	hasta cubrir su totalidad.
	\item \textbf{Formularios:} ofrece interfaces
	para poder completar los	datos que pertenecen
	a la orden de servicio, as\'i puede generarse
	tal documento con los datos completados.
	\item \textbf{Avisos peri\'odicos:} env\'ia
	correos electr\'onicos a los usuarios sobre
	distintos estados en que se encuentra el
	sistema.
	\item \textbf{Notificaciones:} conserva los
	m\'etodos utilizados para avisar a los
	distintos usuarios del sistema sobre las
	situaciones que pueden ocurrir, dentro del
	ecosistema del software.
	\item \textbf{Estad\'istica:} se dedica a los
	c\'alculos estad\'isticos sobre la completitud
	de los legajos y la carga horaria de los
	docentes. Su salida genera los informes
	estad\'isticos solicitados.
	\item \textbf{Ajuste de par\'ametros:} guarda
	cada uno de los par\'ametros ajustables que
	involucran otros m\'odulos. Como salida, se
	devuelve el conjunto de variables globales y
	parametrizables dentro del software.
	\item \textbf{Auditor\'ia:} registra cada una
	de las acciones que se realizan dentro del
	sistema, para controlar los sucesos que vayan
	ocurriendo a lo largo del uso del mismo.
	\item \textbf{Seguridad:} asegura que el
	ingreso de cualquier usuario que intente
	acceder al sistema sea v\'alido, as\'i como
	limitar el contenido disponible para cada rol.
\end{itemize}
\newpage
\subsection*{Gesti\'on de roles}
\begin{itemize}
	\item \textbf{Administrador:} parametriza los
	valores necesarios para que el sistema pueda
	funcionar con flexibilidad, pero sin perder
	las caracter\'isticas deseadas; se trata de
	las operaciones ABM y CRUD, de los
	par\'ametros ajustables, etcétera.
	\item \textbf{Representante T\'ecnico:} tiene
	el poder de adjudicaci\'on a los servicios
	que la universidad realice, sea elegido por
	el Comitente o se apunte al servicio en
	particular; asigna los recursos del servicio,
	y propone los Compromisos y Retribuciones
	de cada servicio que se halle a cargo. Tambi\'en
	es una de las partes que firma una orden de
	servicio.
	\item \textbf{Comitente:} solicita \'ordenes
	de servicio a la universidad, pudiendo elegir
	el Representante T\'ecnico (en caso de desear
	uno en particular); as\'i como decide qu\'e
	acci\'on tomar ante la propuesta de Compromisos
	y Retribuciones de un servicio, sea renegociar,
	aceptar o rechazar la propuesta. Tambi\'en es
	una de las partes que firma una orden de
	servicio.
	\item \textbf{Secretario:} revisa cada una de
	las \'ordenes de servicio y otorga su firma
	final en el documento, pudiendo subirse en
	el sistema. Tambi\'en se encarga de tomar
	decisiones en casos de conflicto por los
	servicios o sus solicitudes.
	\item \textbf{Ayudante:} se encarga de subir
	las \'ordenes de servicio que hayan sido
	firmadas manuscritamente y las facturas que
	corresponden por servicio.
	\item \textbf{Tesorero:} anexa el recibo
	correspondiente a un servicio cobrado.
\end{itemize}
\newpage
\addcontentsline{toc}{section}{Procesos
	automatizados
}
\section*{Procesos automatizados}
\begin{center}
\begin{longtable}{
	| p{3.25cm} | p{5.25cm} | p{7.4cm} |
}
	\hline
	\rowcolor{gray}
	\multicolumn{1}{|p{3.25cm}|}{
		\centering
		\textbf{Denominaci\'on}
	} &
	\multicolumn{1}{p{5.25cm}|}{
		\centering
		\textbf{M\'odulos que intervienen}
	} &
	\multicolumn{1}{p{7.4cm}|}{
		\centering
		\textbf{Descripci\'on del proceso}
	} \\
	\hline
	\endhead
	\multicolumn{1}{|p{3.25cm}|}{
		\raggedleft
		Firma digital
	} &
	\multirowcell{1}{
		\Centerstack{
			\'Ordenes de Servicio \\
			Formularios \\
			Avisos peri\'odicos \\
			Notificaciones \\
			Representante T\'ecnico \\
			Comitente \\
			Secretario
		}
	} &
	\multicolumn{1}{p{7.4cm}|}{
		El sistema administrar\'a la firma
		de \'ordenes de servicios de forma
		digital, cuando el Comitente marque
		que puede firmar documentos
		digitalmente. Este m\'odulo se
		ejecutar\'a cuando se acepte las
		condiciones propuestas por el
		Representante T\'ecnico. Se espera
		que firme el Comitente primero,
		despu\'es el Responsable T\'ecnico
		y por \'ultimo la persona encargada
		de la Secretar\'ia de Extensi\'on.
	} \\
	\hline
	\multicolumn{1}{|p{3.25cm}|}{
		\raggedleft
		Verificaci\'on autom\'atica de
		\'ordenes de servicio firmadas
		de forma manuscrita
	} &
	\multirowcell{1}{
		\Centerstack{
			\'Ordenes de Servicio \\
			Formularios \\
			Avisos peri\'odicos \\
			Notificaciones \\
			Representante T\'ecnico \\
			Comitente \\
			Secretario
		}
	} &
	\multicolumn{1}{p{7.4cm}|}{
		El sistema controlar\'a el documento
		que se suba como orden de servicio,
		cuando el Comitente marque que no
		puede firmar documentos digitalmente.
		Este m\'odulo se ejecutar\'a cuando
		se firme la orden de servicio en papel.
		Se espera que verifique a qu\'e orden
		de servicio pertenece el documento
		subido, la coincidencia de su contenido
		con el formulario rellenado anteriormente
		y la existencia de las 3 firmas en
		el papel.
	} \\
	\hline
	\multicolumn{1}{|p{3.25cm}|}{
		\raggedleft
		Gestor autom\'atico de servicios
	} &
	\multirowcell{1}{
		\Centerstack{
			\'Ordenes de Servicio \\
			Presupuestos \\
			Formularios \\
			Avisos peri\'odicos \\
			Notificaciones \\
			Representante T\'ecnico \\
			Comitente \\
			Secretario
		}
	} &
	\multicolumn{1}{p{7.4cm}|}{
		El sistema controlar\'a el flujo
		de operaci\'on de un servicio,
		realizando acciones autom\'aticamente
		al tener las condiciones adecuadas.
		Este m\'odulo abarcar\'a desde la
		solicitud de servicio hasta la firma de
		su orden (o la cancelaci\'on de alguno
		de los mencionados anteriormente). Se
		espera que anuncie las acciones a realizar
		hacia el actor que corresponda, anule
		una solicitud u orden de servicio tras
		no cumplir el flujo configurado, permita
		redefinir una proposici\'on de
		Compromisos y Retribuciones al paso de
		cierto tiempo y proporcione renegociaci\'on
		a la solicitud de servicio.
	} \\
	\hline
\end{longtable}
\end{center}
\newpage
\addcontentsline{toc}{section}{Estimaci\'on de
	tama\~no por m\'odulo
}
\section*{Estimaci\'on de tama\~no por m\'odulo}
\begin{center}
\begin{longtable}{
	| p{7.95cm} | p{7.95cm} |
}
	\hline
	\rowcolor{gray}
	\multicolumn{1}{|p{7.95cm}|}{
		\centering
		\textbf{M\'odulo}
	} &
	\multicolumn{1}{p{7.95cm}|}{
		\centering
		\textbf{Porcentaje de Participaci\'on /
			Producto
		}
	} \\
	\hline
	\endhead
	\rowcolor{lightgray}
	\textbf{Total} & 100\% \\
	\hline
	\endfoot
	Firma digital & 20\% \\
	\hline
	Verificaci\'on autom\'atica de \'ordenes de
	servicio firmadas de forma manuscrita & 20\% \\
	\hline
	Gestor autom\'atico de servicios & 15\% \\
	\hline
	\'Ordenes de servicio & 10\% \\
	\hline
	Presupuestos & 5\% \\
	\hline
	Cobros & 5\% \\
	\hline
	Formularios & 5\% \\
	\hline
	Avisos peri\'odicos & 5\% \\
	\hline
	Notificaciones & 5\% \\
	\hline
	Estad\'istica & 5\% \\
	\hline
	Gesti\'on de roles & 3\% \\
	\hline
	Ajuste de par\'ametros & 2\% \\
	\hline
\end{longtable}
\end{center}
\newpage
\addcontentsline{toc}{section}{Entorno
	tecnol\'ogico y metodol\'ogico
}
\section*{Entorno tecnol\'ogico y
	metodol\'ogico
}
\begin{center}
\begin{tabular}{ | e | p{10.3cm} | }
	\hline
	\textbf{Lenguajes de programaci\'on} &
	Python - JavaScript \\
	\hline
	\textbf{Framework} & Django \\
	\hline
	\textbf{Arquitectura} &
	Modelo - Plantilla - Vista \\
	\hline
	\textbf{Motor de Base de Datos} &
	PostgreSQL \\
	\hline
	\textbf{Metodolog\'ia seleccionada} &
	Proceso Unificado (UP) \\
	\hline
	\textbf{Tipo de proyecto} &
	Con cliente final \\
	\hline
\end{tabular}
\end{center}
\newpage
\addcontentsline{toc}{section}{Planificaci\'on
	de actividades
}
\section*{Planificaci\'on de actividades}
\begin{center}
\begin{ganttchart}[
	expand chart=\textwidth,
	title/.append style={
		draw=black,
		fill=lightgray
	},
	group/.append style={
		draw=black
	},
	hgrid,
	vgrid={*{6}{draw=none},dotted},
	y unit title=0.5cm,
	y unit chart=0.5cm,
	time slot format=little-endian,
	calendar week text= {\tiny
		{\currentweek}
	},
	title height=1,
	progress label text={},
	bar height=0.4
]{15.05.2023}{17.12.2023}
	\gantttitle[
		title label node/.append style=
			{left=0.1cm and 0.05cm}
	]{A\~no}{0}
	\gantttitlecalendar{year} \\
	\gantttitle[
		title label node/.append style=
			{left=0.1cm and 0.05cm}
	]{Mes}{0}
	\gantttitlecalendar{month} \\
	\gantttitle[
		title label node/.append style=
			{left=0.1cm and 0.05cm}
	]{Semana}{0}
	\gantttitlecalendar{week=20} \\
	\ganttgroup[progress=0]
			{Requerimientos}
			{15.05.2023}{30.06.2023} \\
	\ganttgroup[progress=0]
			{An\'alisis}
			{19.06.2023}{21.07.2023} \\
	\ganttgroup[progress=0]
			{Dise\~no}
			{19.06.2023}{18.08.2023} \\
	\ganttgroup[progress=0]
			{Implementaci\'on}
			{24.07.2023}{10.11.2023} \\
	\ganttgroup[progress=0]
			{Pruebas}
			{07.08.2023}{24.11.2023} \\
	\ganttgroup[progress=0]
			{Implantaci\'on}
			{13.11.2023}{24.11.2023} \\
	\ganttgroup[progress=0]
			{Capacitaci\'on}
			{20.11.2023}{24.11.2023} \\
	\ganttgroup[progress=0]
			{Manuales}
			{24.07.2023}{01.12.2023} \\
	\ganttbar[progress=0]
			{Planificaci\'on}
			{15.05.2023}{26.05.2023} \\
	\ganttbar[progress=0]
			{Factibilidad}
			{29.05.2023}{30.06.2023} \\
	\ganttbar[progress=0]
			{Relevamiento}
			{05.06.2023}{16.06.2023} \\
	\ganttbar[progress=0]
			{Requisitos}
			{12.06.2023}{23.06.2023} \\
	\ganttbar[progress=0]
			{Casos de uso}
			{12.06.2023}{30.06.2023} \\
	\ganttbar[progress=0]
			{Modelo de dominio}
			{19.06.2023}{07.07.2023} \\
	\ganttbar[progress=0]
			{Secuencias de sistema}
			{26.06.2023}{14.07.2023} \\
	\ganttbar[progress=0]
			{Contratos}
			{03.07.2023}{21.07.2023} \\
	\ganttbar[progress=0]
			{Casos de uso reales}
			{19.06.2023}{07.07.2023} \\
	\ganttbar[progress=0]
			{Secuencias de dise\~no}
			{26.06.2023}{14.07.2023} \\
	\ganttbar[progress=0]
			{Diagrama de clases}
			{03.07.2023}{21.07.2023} \\
	\ganttbar[progress=0]
			{Dise\~no de entradas}
			{10.07.2023}{28.07.2023} \\
	\ganttbar[progress=0]
			{Dise\~no de salidas}
			{17.07.2023}{04.08.2023} \\
	\ganttbar[progress=0]
			{Dise\~no de interfaces}
			{24.07.2023}{11.08.2023} \\
	\ganttbar[progress=0]
			{Dise\~no de base de datos}
			{31.07.2023}{18.08.2023} \\
	\ganttbar[progress=0]
			{50\% de implementaci\'on}
			{24.07.2023}{15.09.2023} \\
	\ganttbar[progress=0]
			{75\% de implementaci\'on}
			{18.09.2023}{13.10.2023} \\
	\ganttbar[progress=0]
			{100\% de implementaci\'on}
			{16.10.2023}{10.11.2023} \\
	\ganttbar[progress=0]
			{50\% de pruebas}
			{07.08.2023}{29.09.2023} \\
	\ganttbar[progress=0]
			{75\% de pruebas}
			{02.10.2023}{28.10.2023} \\
	\ganttbar[progress=0]
			{100\% de pruebas}
			{31.10.2023}{24.11.2023} \\
	\ganttbar[progress=0]
			{Implantaci\'on}
			{13.11.2023}{24.11.2023} \\
	\ganttbar[progress=0]
			{Capacitaci\'on}
			{20.11.2023}{24.11.2023} \\
	\ganttbar[progress=0]
			{Manuales}
			{24.07.2023}{01.12.2023} \\
	\ganttbar[progress=0]
			{Ambientaci\'on personal}
			{20.11.2023}{01.12.2023} \\
	\ganttbar[progress=0]
			{Pr\'actica de exposici\'on}
			{04.12.2023}{09.12.2023}
\end{ganttchart}
\end{center}
\newpage
\begin{center}
\begin{longtable}{
	| p{5.3cm} | p{5.3cm} | p{5.3cm} |
}
	\hline
	\rowcolor{lightgray}
	\multicolumn{1}{|p{5.3cm}|}{
		\centering
		\textbf{Actividad}
	} &
	\multicolumn{1}{p{5.3cm}|}{
		\centering
		\textbf{Fecha de inicio}
	} &
	\multicolumn{1}{p{5.3cm}|}{
		\centering
		\textbf{Fecha de finalizaci\'on}
	} \\
	\hline
	\endhead
 	Planificaci\'on &
 	15/05/2023 & 26/05/2023 \\
	\hline
 	Factibilidad &
 	29/05/2023 & 30/06/2023 \\
	\hline
 	Relevamiento &
 	05/06/2023 & 16/06/2023 \\
	\hline
 	Requisitos &
 	12/06/2023 & 23/06/2023 \\
	\hline
 	Casos de uso &
 	12/06/2023 & 30/06/2023 \\
	\hline
 	Modelo de dominio &
 	19/06/2023 & 07/07/2023 \\
	\hline
 	Secuencias de sistema &
 	26/06/2023 & 14/07/2023 \\
	\hline
 	Contratos &
 	03/07/2023 & 21/07/2023 \\
	\hline
 	Casos de uso reales &
 	19/06/2023 & 07/07/2023 \\
	\hline
 	Secuencias de dise\~no &
 	26/06/2023 & 14/07/2023 \\
	\hline
 	Diagrama de clases &
 	03/07/2023 & 21/07/2023 \\
	\hline
 	Dise\~no de entradas &
 	10/07/2023 & 28/07/2023 \\
	\hline
 	Dise\~no de salidas &
 	17/07/2023 & 04/08/2023 \\
	\hline
 	Dise\~no de interfaces &
 	24/07/2023 & 11/08/2023 \\
	\hline
 	Dise\~no de base de datos &
 	31/07/2023 & 18/08/2023 \\
	\hline
 	50\% de implementaci\'on &
 	24/07/2023 & 15/09/2023 \\
	\hline
 	75\% de implementaci\'on &
 	18/09/2023 & 13/10/2023 \\
	\hline
 	100\% de implementaci\'on &
 	16/10/2023 & 10/11/2023 \\
	\hline
 	50\% de pruebas &
 	07/08/2023 & 29/09/2023 \\
	\hline
 	75\% de pruebas &
 	02/10/2023 & 28/10/2023 \\
	\hline
 	100\% de pruebas &
 	31/10/2023 & 24/11/2023 \\
	\hline
 	Implantaci\'on &
 	13/11/2023 & 24/11/2023 \\
	\hline
 	Capacitaci\'on &
 	20/11/2023 & 24/11/2023 \\
	\hline
 	Manuales &
 	24/07/2023 & 01/12/2023 \\
	\hline
 	Ambientaci\'on personal &
 	20/11/2023 & 01/12/2023 \\
	\hline
 	Pr\'actica de exposici\'on &
 	04/12/2023 & 09/12/2023 \\
	\hline
\end{longtable}
\end{center}
