\section[Definici\'on]{Informaci\'on del proyecto}
\normalsize{ \indent
Sobre el proyecto se obtiene la siguiente informaci\'on:
}
\begin{center}
\begin{tabular}{ p{5.5cm} p{10cm} }
	\textbf{Organizaci\'on} & Alumnado de M\'odulo Ap\'ostoles \\
	\textbf{Proyecto} & GesServOrConv (Gestor de Servicios por
		Orden o Convenio) \\
	\textbf{Fecha de preparaci\'on} & 6 de mayo de 2024 \\
	\textbf{Cliente} & Secretar\'ia de Extensi\'on y
		Vinculaci\'on Tecnol\'ogica de la FCEQyN de la UNaM \\
	\textbf{Patrocinador (Sponsor)} & FCEQyN (UNaM) \\
	\textbf{Gerente/L\'ider de Proyecto} & Antoniak, Rodrigo
		Lionel
\end{tabular}
\end{center}
\section[Resumen]{Resumen ejecutivo}
\normalsize{ \indent
La Facultad de Ciencias Exactas, Qu\'imicas y Naturales de la
Universidad Nacional de Misiones desea informatizar el proceso
de los servicios que realiza a trav\'es de un sistema software.
Se ha intentado llevar a cabo antes de este proyecto y no
se pudo concretar con \'exito, por lo que se aspira a llegar
a terminar correctamente.
}
\newline
\normalsize{ \indent
Al tener requisitos muy espec\'ificos, es necesario que el
sistema sea personalizado al procedimiento pretendido; as\'i
mismo, al no contar con muchos recursos econ\'omicos, tener
al l\'ider del proyecto trabajando \textit{ad honorem}
resulta ideal.
}
\newline
\normalsize{ \indent
Al tener requisitos muy espec\'ificos, es necesario que el
sistema sea personalizado al procedimiento pretendido; as\'i
mismo, al no contar con muchos recursos econ\'omicos, tener
al l\'ider del proyecto trabajando \textit{ad honorem}
resulta ideal. As\'i tambi\'en, se debe evitar el uso de CDNs
y bases de datos en la nube.
}
\newline
\normalsize{ \indent
Al tener requisitos muy espec\'ificos, es necesario que el
sistema sea personalizado al procedimiento pretendido; as\'i
mismo, al no contar con muchos recursos econ\'omicos, tener
al l\'ider del proyecto trabajando \textit{ad honorem}
resulta ideal. As\'i tambi\'en, se debe evitar el uso de CDNs
y bases de datos en la nube.
}
\section[Antecedentes]{Antecedentes del proyecto}
\normalsize{ \indent
Primero, los factores que justifican la realizaci\'on del proyecto
son la ausencia de informatizaci\'on para las \'ordenes de servicio
(si bien, estos documentos se realizan con un paquete ofim\'atico;
no hay un sistema que automatice el proceso) y el fracaso de un
intento de realizaci\'on del software con otros recursos humanos.
De ah\'i, el estudio de factibilidad es impulsado por las posibles
opciones que existan, la predisposici\'on del cliente a distintas
propuestas y la necesidad de decidir si comprar o desarrollar
software.
}
\newline
\normalsize{ \indent
Adicionalmente, la Secretar\'ia de Extensi\'on y Vinculaci\'on
Tecnol\'ogica fue quien propuso el proyecto, aunque el l\'ider del
proyecto se acerc\'o a preguntar sobre alg\'un problema que ten\'ia
el ente.
}
\section[Contexto]{El proyecto y su contexto}
\normalsize{ \indent
La Facultad de Ciencias Exactas, Qu\'imicas y Naturales de la
Universidad Nacional de Misiones desea informatizar el proceso
de los servicios que realiza a trav\'es de un sistema software.
Los Comitentes son los que solicitan los servicios y las Unidades
Ejecutoras llevan a cabo los mismos, donde los Responsables
T\'ecnicos est\'an a cargo de liderar las unidades en que
tengan ese rol.
}
\newline
\normalsize{ \indent
En cada servicio, se pueden encontrar varios Comitentes y
Responsables T\'ecnicos; por lo que el sistema debe soportar
una forma de acuerdo dentro de cada parte y entre ellas sobre
los participantes del servicio. Tras acordar a las personas
anteriormente mencionadas, se debe acordar los compromisos
y retribuciones econ\'omicas entre Comitentes y Responsables
T\'ecnicos.
}
\newline
\normalsize{ \indent
El siguiente paso, depender\'a del tipo de servicio a seguir.
Por un lado, se encuentran los servicios por orden del mismo;
donde habr\'a un documento que constata el acuerdo de las
partes, en conjunto con la aprobaci\'on de quien se encargue
de la Secretar\'ia de Extensi\'on y Vinculaci\'on Tecnol\'ogica.
Para este tipo de servicios, el sistema debe poder armar el
documento desde una plantilla; habilitando la posibilidad
de ser firmado digitalmente o de forma manuscrita, donde
este \'ultimo se subir\'a como un archivo escaneado y
se validar\'a con un proceso automatizado.
}
\newline
\normalsize{ \indent
Por otro lado, se hallan los servicios realizados por convenios;
donde habr\'a una resoluci\'on del Consejo Directivo que
refleje el trato realizado y s\'olo se pretende firmar
de manera manuscrita. Considerando la naturaleza de
estos convenios, se prefiere vincular directamente la
resoluci\'on al servicio a realizar con un escaneo;
sin necesidad de comprobar su validez, ya que proviene
de una entidad superior.
}
\newline
\normalsize{ \indent
Sin importar el tipo de servicio, ambos tendr\'an que ser
monitoreados durante su realizaci\'on; esto es, registrar
la facturaci\'on y los recibos del servicio, as\'i como
guardar los progresos de ejecuci\'on.
}
\newline
\normalsize{ \indent
Para cada etapa del sistema, se deber\'a encontrar la
posibilidad de cancelar el servicio con los permisos
correspondientes; as\'i como la capacidad de suspender
autom\'aticamente sobre la solicitud o la orden del
servicio (no aplica sobre los convenios, ya que el
Consejo Directivo no act\'ua sobre el software).
}
\newline
\normalsize{ \indent
Respecto al contexto del proyecto, el l\'ider es alumno
de la facultad y posee cercan\'ia con los directivos
de la entidad; lo cual, facilita la comunicaci\'on con
los que se requiera hacer preguntas. Tomando en cuenta el
di\'alogo en las entrevistas, puede afirmarse que hubo
acuerdo en la visi\'on de cada entrevistado.
}
\section[Alcance]{Alcance del estudio de factibilidad}
\normalsize{ \indent
El alcance pretendido para este estudio es el conocimiento de
los pasos a seguir para poder gestionar los servicios.
Principalmente, se analizar\'an distintas opciones por los
costos para elegir; considerando que debe cumplirse la
condici\'on de poder usar el personal ya contratado de la
facultad con la posible soluci\'on y debe estar instaurado
como un sistema en dominio propio, no como un servicio.
}
\section[An\'alisis]{An\'alisis de Factibilidad}
\subsection{Factibilidad t\'ecnica}
\normalsize{ \indent
En la b\'usqueda de las distintas soluciones, se ha encontrado
distintos candidatos para conformar el software final. En
principio, se descartan las soluciones que sean como
servicios; en b\'usqueda de implantar la totalidad del
sistema \textit{in situ}.
}
\newline
\normalsize{ \indent
De ah\'i, tambi\'en se descarta la inclusi\'on de otros
sistemas ya existentes; considerando que el contexto del
sistema es suficientemente propio y no est\'a representado
por ning\'un software m\'as gen\'erico. Por lo tanto,
las opciones restantes son de quienes proporcionan un
sistema hecho a medida; en donde la elecci\'on del
desarrollador depender\'a de otros factores.
}
\subsection{Factibilidad econ\'omica}
\normalsize{ \indent
Este factor es esencial, en consecuencia del inter\'es
de la facultad en ahorrar los recursos econ\'omicos lo
m\'as posible; tomando en cuenta que el proyecto se
har\'ia a medida solamente para el cliente en cuesti\'on.
}
\newline
\normalsize{ \indent
Dicho lo anterior, se debe evitar el uso de CDNs y bases
de datos en la nube. As\'i mismo, el desarrollador del
sistema debe cobrar una cantidad asequible para la
facultad; por lo que el l\'ider del proyecto se ofrece
a realizar el sistema \textit{ad honorem}.
}
\subsection{Factibilidad legal}
\normalsize{ \indent
Tomando en cuenta que se pretende hacer uso de un sistema
experto para las \'ordenes de servicio manuscritas, debe
considerarse las recomendaciones para el manejo de una
inteligencia artificial; cuya documentaci\'on se encuentra
en la disposici\'on 2/2023, que se encuentra en el
siguiente enlace:
\url{https://www.argentina.gob.ar/normativa/nacional/disposici\%C3\%B3n-2-2023-384656/texto}
}
\newline
\normalsize{ \indent
De ah\'i, hay que hacer \'enfasis en la secci\'on 3.1 de
las Recomendaciones para una Inteligencia Artificial Fiable;
donde establece que no se puede atribuir responsabilidades
de las acciones que ejecuta a una inteligencia artificial.
Es decir, m\'as all\'a de implementar la caracter\'istica
citada en el p\'arrafo anterior; es responsabilidad de
la persona que suba el documento sobre la validez de la
misma.
}
\newline
\normalsize{ \indent
Adicionalmente, debe considerarse el manejo de los datos
personales; cuya protecci\'on est\'a normalizada en la ley
25326, que se encuentra en el siguiente enlace:
\url{https://www.argentina.gob.ar/normativa/nacional/ley-25326-64790/actualizacion}.
En ella, se destaca el art\'iculo 10 que obliga el secreto
profesional de los datos personales a quienes traten con
ellos (incluso despu\'es de terminar la soluci\'on completa).
}
\subsection{Factibilidad de recursos}
\normalsize{ \indent
Desde el inicio del proyecto, la universidad ha buscado realizar
operaciones con los recursos existentes y evitar el uso de
otros adicionales; por lo tanto, el sistema debe utilizar los
recursos humanos en disposici\'on y trabajar con el hardware
y software que provea la facultad.
}
\newline
\normalsize{ \indent
Desde el punto de vista inform\'atico, la sede central de la
FCEQyN cuenta con los servidores a disposici\'on; cuyos sistemas
implantados se prefieren implantar como m\'aquinas virtuales
ante contenedores. La solvencia del dominio web estar\'a fuera
de las tareas de los implementadores, por lo que no est\'a
abarcado.
}
\subsection{Factibilidad de mercado}
\normalsize{ \indent
Como se ha citado en la factibilidad t\'ecnica, las opciones
viables para este tipo de soluciones se encuentra en quienes
provean software personalizado ante las necesidades del que
lo requiera. Entonces, el producto a desarrollar se hallar\'ia
compitiendo con lo que ofrezcan otros programadores sobre
sistemas a medida.
}
\newline
\normalsize{ \indent
Para el cliente final que se encuentra en este documento,
no se requiere estrategias de mercadeo; tomando en
consideraci\'on el trabajo \textit{ad honorem}. Para
pr\'oximas implantaciones a otros clientes, se plantear\'a
el cobro del sistema; aunque no ser\'a explayado en esta
factibilidad, ya que est\'a fuera del contexto de este
proyecto (al ser a medida, otras entidades tendr\'an
requerimientos distintos).
}
\subsection{Factibilidad operacional}
\normalsize{ \indent
Primero, todos los Comitentes deben estar de acuerdo entre
si para pertenecer a una solicitud de servicio. Tras
este proceso, tienen la opci\'on de proponer los
Responsables T\'ecnicos o que se postulen ellos mismos
para dirigir la ejecuci\'on del servicio; donde cada
Comitente puede elegir el candidato y/o aceptar uno.
}
\newline
\normalsize{ \indent
Una vez acordados los participantes directos del servicio,
se debe formalizar el acuerdo; ya sea por orden o convenio.
En caso de ser orden, los Comitentes y Responsables
T\'ecnicos negocian los compromisos y retribuciones
econ\'omicas hasta estar de acuerdo; entonces, se firma
el documento de la orden con cada Comitente y Responsable
T\'ecnico, en conjunto con la persona encargada de la
Secretar\'ia de Extensi\'on y Vinculaci\'on Tecnol\'ogica.
Si resulta elegirse un convenio, el Consejo Directivo
decidir\'a sobre los t\'erminos y condiciones; donde
la comunicaci\'on ser\'a externa, hasta que se concrete
la firma de resoluci\'on para el servicio.
}
\newline
\normalsize{ \indent
Con el documento necesario firmado, ha de facturarse el
servicio y se obtiene el comprobante de tal acci\'on;
para posteriormente registrar cada uno de los pagos del
servicio con el recibo que corresponda. Al mismo tiempo,
los Responsables T\'ecnicos registran progresos sobre
los servicios que tengan a cargo hasta completarlos.
}
\newline
\normalsize{ \indent
Observando que se tiene el personal adecuado para todo el
proceso, no se necesita alterar la metodolog\'ia de
los pasoso actuales; sino, incluir las tecnolog\'ias
del software e incorporar su uso en el contexto de los
servicios.
}
\subsection{Factibilidad de tiempo}
\normalsize{ \indent
Considerando que la competencia no cubre las necesidades
de la universidad con un producto ya hecho y que el costo
econ\'omico que propone el l\'ider del proyecto es nulo, no
hay un riesgo aparente del tiempo frente a otros que puedan
hacer la misma tarea en menor tiempo. Sin embargo, no hay que
olvidarse que uno se compromete con la entrega del software;
por lo tanto, se debe seguir los tiempos previamente tratados
con el cliente final.
}
\section{Recomendaciones}
\normalsize{ \indent
Gracias a que el desarrollador se ofrece a realizar el sistema
sin costo econ\'omico, es preferible tomar tal opci\'on; tras
cumplir las prioridades de la universidad respecto al
software que necesitan.
}
\newline
\normalsize{ \indent
Cuando se siga con otros objetivos que no se abarcan en el
alcance actual, es recomendable solicitar al mismo desarrollador;
es la persona quien conoce mejor el sistema, por lo que se
tendr\'a menor dificultad para expandir el software pretendido.
}
