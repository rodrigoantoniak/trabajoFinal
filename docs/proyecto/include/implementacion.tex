\section{Bases de datos}
\normalsize{\indent
Para la administraci\'on de datos, se
ha utilizado PostgreSQL como almacenamiento
principal y Redis como complemento; donde
este \'ultimo abarca el control de
sesiones, la distribuci\'on de mensajes
as\'incrono, rate limit, cach\'e y back-end
de gesti\'on de tareas. Respecto al
motor de base de datos SQL, se ha utilizado
las extensiones hstore (para utilizar
diccionarios), btree\_gin (para el indizado
de palabras clave en forma de \'arbol B,
destinado a b\'usquedas avanzadas) y
unaccent (para remover acentos, utilizado
en b\'usquedas).
}
\section{Back-end}
\normalsize{\indent
Para el desarrolo de la capa entre la
interfaz de usuario y la base de datos,
se ha utilizado el framework Django 4.2
(por consecuente, el lenguaje de
programaci\'on es Python, en su versi\'on
3.11) junto con el manejador de tareas
Celery 5.4 (para a gesti\'on de tareas
peri\'odicas, crons, entre otros).
Como complementos a Django, se utiliza
Channels para utilizar WebSockets
(destinado a notificaciones) a trav\'es
del servidor Daphne. En adici\'on,
se hace uso de otras librer\'ias para
funcionalidades m\'as espec\'ificas:
}
\begin{itemize}
	\item \textbf{Reportlab:} sirve
	para obtener los reportes en el
	sistema.
	\item \textbf{Endesive:} sirve para
	efectuar las firmas digitales.
	\item \textbf{OpenCV 4:} sirve para
	la manipulaci\'on de im\'agenes en
	la validaci\'on de documentos
	firmados a mano.
	\item \textbf{Tesseract 5:} sirve
	para leer el contenido de texto en
	la validaci\'on de documentos
	firmados a mano.
	\item \textbf{Matplotlib 3:} sirve
	para obtener las estad\'isticas
	en el sistema.
\end{itemize}
\section{Front-end}
\normalsize{\indent
Para la interfaz de usuario, se ha
utilizado el motor de plantillas de
Django por defecto; junto con el
framework CSS Bootstrap 5.3 y la
librer\'ia de JavaScript HTMX 1.9.
Respecto al marco de trabajo, se
posee el bundle completo; mientras
que de la librer\'ia, se agrega las
extensiones de WebSocket y Preload.
Adicionalmente, se ocupa las librer\'ias
de JavaScript jQuery 3.6 y Select2
(versi\'on 4); donde el primero es
una dependencia del segundo, adem\'as
de utilizarse en determinados casos
para facilitar la resoluci\'on
de conflictos con algunos componentes
espec\'ificos en Bootstrap.
}
\section{Entorno de trabajo}
\normalsize{\indent
Para la codificaci\'on del software,
se ha utilizado VSCodium en el
sistema operativo OpenSUSE
Tumbleweed.
}