\normalsize{ \indent
En esta secci\'on, se presenta la organizaci\'on
en fases e iteraciones, y el calendario del
proyecto.
}
\section[Fases]{Plan de las Fases}
\normalsize{ \indent
El desarrollo se llevar\'a a cabo en base a fases
con una iteraci\'on en cada una de ellas. La
siguiente tabla muestra la distribuci\'on de
tiempos y el n\'umero de iteraciones de cada fase.
}
\begin{table}[h!]
	\begin{center}
	\begin{tabular}{ | p{3cm} | p{4cm} | p{3cm} | }
		\hline
		\rowcolor{lightgray}
		\hfil \textbf{Fase} &
		\hfil \textbf{N\textordmasculine \ de Iteraciones} &
		\hfil \textbf{Duraci\'on} \\ 
		\hline
		\textit{Incepci\'on} & 1 & 6 semanas \\
		\hline
		\textit{Elaboraci\'on} & 1 & 5 semanas \\
		\hline
		\textit{Construcci\'on} & 1 & 11 semanas \\
		\hline
		\textit{Transici\'on} & 1 & 6 semanas \\
		\hline
	\end{tabular}
	\end{center}
	\caption{Fases de la planificaci\'on}
\end{table}
\clearpage
\normalsize{ \indent
Los hitos que marcan el final de cada fase se describen
en la siguiente tabla:
}
\begin{table}[h!]
	\begin{center}
	\begin{tabular}{ | p{3cm} | p{12.5cm} | }
		\hline
		\rowcolor{lightgray}
		\hfil \textbf{Fase} &
		\hfil \textbf{Hito} \\
		\hline
		\textit{Incepci\'on} &
		Se presentar\'a el plan de proyecto, junto con
		la planificaci\'on de la primer iteraci\'on.
		Se identificar\'a a los casos de usos principales
		del sistema realizando una breve descripci\'on del
		mismo. Se realizar\'a un estudio para hallar
		firmadores digitales que puedan ser utilizados en
		una aplicaci\'on web.
		\\
		\hline
		\textit{Elaboraci\'on} &
		Se analizar\'a los requisitos y se desarrollar\'a
		un prototipo de arquitectura (incluyendo las partes
		m\'as relevantes y/o cr\'iticas del sistema). Al
		final de esta fase, todos los casos de uso
		correspondientes a requisitos deber\'an estar
		analizados y dise\~nados. Se comienza la
		elaboraci\'on de material de apoyo al usuario. La
		revisi\'on y aceptaci\'on del prototipo de la
		arquitectura del sistema marca el final de esta fase.
		\\
		\hline
		\textit{Construcci\'on} &
		Se construir\'a el producto, al cual se le aplicar\'an
		las pruebas necesarias para validar y verificar el
		software. Se realizar\'a la mayor\'ia del material de
		apoyo al usuario. El hito que marca el fin de esta fase
		es la versi\'on que tenga la capacidad operacional
		completa del producto (habi\'endose evaluado la
		mayor\'ia del sistema, pero no en su totalidad), lista
		para ser entregada a los usuarios para pruebas de
		aceptaci\'on.
		\\
		\hline
		\textit{Transici\'on} &
		Se preparará una release para distribuci\'on,
		asegurando una implantaci\'on adecuada. El hito
		que marca el fin de esta fase incluye la entrega
		de toda la documentaci\'on del proyecto con los
		manuales de instalaci\'on y todo el material de
		apoyo al usuario y el empaquetamiento del producto.
		\\
		\hline
	\end{tabular}
	\end{center}
	\caption{Hitos por cada fase de la planificaci\'on}
\end{table}
\clearpage
\section[Entrevistas]{Planificaci\'on de las Entrevistas}
\normalsize{ \indent
Primero, se investig\'o sobre los antecedentes del colegio;
para lo cual, fue necesario realizar una primera entrevista
para averiguar detalles hist\'oricos del manejo de
informaci\'on. Teniendo en cuenta que el problema inicial
resultaba ser muy reducido para el contexto de Trabajo
Final/Proyecto de Software, se recolect\'o m\'as datos de
la situaci\'on alrededor de servicios. Toda la informaci\'on
respecto a este asunto se encuentra en la descripci\'on del
escenario.
}
\newline
\normalsize{ \indent
El paso siguiente fue establecer los objetivos de las
entrevistas (se han llevado a cabo tres), las cuales tienen
enfoques distintos por la informaci\'on que se manejaba de
los servicios y por el asentamiento del alcance de la
soluci\'on. Para las primeras dos entrevistas, se propuso
como objetivos: 
}
\begin{itemize}
	\item Conocer los antecedentes de la universidad con las
	\'ordenes y realizaci\'on de servicios.
	\item Informarse sobre los problemas que posee la
	universidad respecto a los servicios. Adem\'as, constatar
	si existe alg\'un tipo de soluci\'on sugerida por los
	futuros clientes del sistema.
	\item Comprobar si las sugerencias generadas por el
	entrevistador para el sistema son viables para los
	entrevistados. Es decir, si las ideas de la posible
	soluci\'on son convenientes para el software que se
	genere (al menos, seg\'un los clientes del futuro software).
	\item Recolectar datos sobre los procedimientos para
	servicios.
\end{itemize}
\ \newline
\normalsize{ \indent
Para la entrevista m\'as reciente, con un alcance dimensionado;
han surgido nuevos objetivos de la entrevista, los cuales son:
}
\begin{itemize}
	\item Investigar la clasificaci\'on de los servicios seg\'un
	el origen.
	\item Identificar los derechos que los usuarios finales tienen
	sobre el software.
	\item Recolectar datos con objeto de analizar las condiciones
	necesarias para cada etapa del servicio.
	\item Establecer las reglas a cumplir sobre la solicitud de
	servicios, la firma de documentaci\'on sobre lo solicitado y
	el servicio por si mismo (tanto por orden como por convenios).
	\item Definir el car\'acter de la soluci\'on frente a las
	distintas situaciones que se puedan presentar.
\end{itemize}
\ \newline
\normalsize{ \indent
A continuaci\'on, deb\'ia decidirse a las personas que ser\'ian
entrevistadas; de todas formas, la universidad estaba
familiarizada con la realizaci\'on de sistemas y ten\'ia
conocimiento sobre el personal adecuado para ello. Por lo tanto,
se ha entrevistado a la Secretaria de Extensi\'on y Vinculaci\'on
Tecnol\'ogica y personal auxiliar de la secretar\'ia anteriormente
mencionada.
}
\newline
\normalsize{ \indent
Luego, la preparaci\'on de los entrevistados estuvo a cargo de
la Secretaria de Extensi\'on y Vinculaci\'on Tecnol\'ogica; ya
que es la encargada de la secretar\'ia a quien se destina la
soluci\'on y est\'a m\'as adentrada en la gesti\'on de los
procedimientos involucrados. Con tal preparaci\'on, se ha
recibido un ejemplo de orden de servicio como anexo.
}
\newline
\normalsize{ \indent
Por \'ultimo, fue momento de tomar una postura ante el tipo de
preguntas y su estructura. Sobre el primer tema mencionado en
este p\'arrafo, se distingue el uso de preguntas abiertas,
cerradas y sondeos para todas las entrevistas. Por otra parte,
la estructura elegida para cada entrevista fue distinta; en
gran parte por la informaci\'on en el momento que se realiz\'o
cada una.
}
\newline
\normalsize{ \indent
En la primera, se opt\'o por una forma de embudo; comenzando
a indagar sobre los servicios por si mismos y las \'ordenes
de ellos, los m\'etodos de comunicaci\'on entre los usuarios
para acordar un servicio. Después, se hacen las preguntas
m\'as cerradas para concretar en un contexto definido de la
situaci\'on; averiguando las demandas de los clientes del
futuro software. Al final, se concreta en una forma inicial
de la soluci\'on.
}
\newline
\normalsize{ \indent
En la segunda, se opt\'o por una forma de diamante; ubicando
al principio preguntas espec\'ificas sobre los procesos
automatizados. Por siguiente, se indag\'o sobre el circuito
del procedimiento de servicio por orden; para as\'i, ir
especificando sobre cada uno de esos pasos.
}
\newline
\normalsize{ \indent
En la tercera, se opt\'o por una forma de pir\'amide;
situando las especificaciones sobre los seguros y
aseguradoras, respecto a los servicios. De ah\'i, en
caso de que tales caracter\'isticas no puedan ser
suficientes para Trabajo Final/Proyecto de Software;
se indagar\'a sobre otros circuitos disponibles sobre
los servicios, revisando partes en particular hasta
generalizar el alcance del sistema definitivo.
}
\newline
\normalsize { \indent
Para terminar, cabe mencionar que el uso de sondeos resulta
dif\'icil ubicar en momentos espec\'ificos de la entrevista;
este tipo de preguntas se realiza respecto a cierto tipo de
respuestas a una cuesti\'on anterior, por lo que el
entrevistador debe prestar atenci\'on al utilizar los mismos
en tiempo real. Igualmente, se tiene una noci\'on de las
situaciones clave gracias a la experiencia.
}
\clearpage
\section[Actividades]{Planificaci\'on de las Actividades}
\begin{figure}[h!]
	\begin{center}
	\begin{ganttchart}[
		expand chart=\textwidth,
		title/.append style={
			draw=black,
			fill=lightgray
		},
		group/.append style={
			draw=black
		},
		hgrid,
		vgrid={*{6}{draw=none},dotted},
		y unit title=0.5cm,
		y unit chart=0.5cm,
		time slot format=little-endian,
		calendar week text= {\tiny
			{\currentweek}
		},
		title height=1,
		progress label text={},
		bar height=0.4
	]{22.04.2024}{20.12.2024}
		\gantttitle[
			title label node/.append style=
				{left=0.1cm and 0.05cm}
		]{A\~no}{0}
		\gantttitlecalendar{year} \\
		\gantttitle[
			title label node/.append style=
				{left=0.1cm and 0.05cm}
		]{Mes}{0}
		\gantttitlecalendar{month} \\
		\gantttitle[
			title label node/.append style=
				{left=0.1cm and 0.05cm}
		]{Semana}{0}
		\gantttitlecalendar{week=17} \\
		\ganttgroup[progress=0]
				{Requerimientos}
				{06.05.2024}{14.06.2024} \\
		\ganttgroup[progress=0]
				{An\'alisis}
				{27.05.2024}{28.06.2024} \\
		\ganttgroup[progress=0]
				{Dise\~no}
				{27.05.2024}{26.07.2024} \\
		\ganttgroup[progress=0]
				{Implementaci\'on}
				{10.06.2024}{27.09.2024} \\
		\ganttgroup[progress=0]
				{Pruebas}
				{24.06.2024}{11.10.2024} \\
		\ganttgroup[progress=0]
				{Implantaci\'on}
				{30.09.2024}{26.10.2024} \\
		\ganttgroup[progress=0]
				{Capacitaci\'on}
				{14.10.2024}{08.11.2024} \\
		\ganttgroup[progress=0]
				{Manuales}
				{24.06.2024}{08.11.2024} \\
		\ganttbar[progress=0]
				{Planificaci\'on}
				{22.04.2024}{03.05.2024} \\
		\ganttbar[progress=0]
				{Factibilidad}
				{06.05.2024}{07.06.2024} \\
		\ganttbar[progress=0]
				{Relevamiento}
				{13.05.2024}{24.05.2024} \\
		\ganttbar[progress=0]
				{Requisitos}
				{20.05.2024}{07.06.2024} \\
		\ganttbar[progress=0]
				{Casos de uso}
				{20.05.2024}{14.06.2024} \\
		\ganttbar[progress=0]
				{Modelo de dominio}
				{27.05.2024}{14.06.2024} \\
		\ganttbar[progress=0]
				{Secuencias de sistema}
				{03.06.2024}{21.06.2024} \\
		\ganttbar[progress=0]
				{Contratos}
				{10.06.2024}{28.06.2024} \\
		\ganttbar[progress=0]
				{Casos de uso reales}
				{27.05.2024}{14.06.2024} \\
		\ganttbar[progress=0]
				{Secuencias de dise\~no}
				{03.06.2024}{21.06.2024} \\
		\ganttbar[progress=0]
				{Diagrama de clases}
				{10.06.2024}{28.06.2024} \\
		\ganttbar[progress=0]
				{Dise\~no de entradas}
				{17.06.2024}{05.07.2024} \\
		\ganttbar[progress=0]
				{Dise\~no de salidas}
				{24.06.2024}{12.07.2024} \\
		\ganttbar[progress=0]
				{Dise\~no de interfaces}
				{01.07.2024}{19.07.2024} \\
		\ganttbar[progress=0]
				{Dise\~no de base de datos}
				{08.07.2024}{26.07.2024} \\
		\ganttbar[progress=0]
				{50\% de implementaci\'on}
				{10.06.2024}{19.07.2024} \\
		\ganttbar[progress=0]
				{75\% de implementaci\'on}
				{22.07.2024}{23.08.2024} \\
		\ganttbar[progress=0]
				{100\% de implementaci\'on}
				{26.08.2024}{27.09.2024} \\
		\ganttbar[progress=0]
				{50\% de pruebas}
				{24.06.2024}{05.08.2024} \\
		\ganttbar[progress=0]
				{75\% de pruebas}
				{08.08.2024}{06.09.2024} \\
		\ganttbar[progress=0]
				{100\% de pruebas}
				{09.09.2024}{11.10.2024} \\
		\ganttbar[progress=0]
				{Implantaci\'on}
				{30.09.2024}{26.10.2024} \\
		\ganttbar[progress=0]
				{Capacitaci\'on}
				{14.10.2024}{08.11.2024} \\
		\ganttbar[progress=0]
				{Manuales}
				{24.06.2024}{08.11.2024} \\
		\ganttbar[progress=0]
				{Ambientaci\'on personal}
				{11.11.2024}{22.11.2024} \\
		\ganttbar[progress=0]
				{Pr\'actica de exposici\'on}
				{25.11.2024}{07.12.2024}
	\end{ganttchart}
	\end{center}
	\caption{Diagrama de Gantt}
\end{figure}
