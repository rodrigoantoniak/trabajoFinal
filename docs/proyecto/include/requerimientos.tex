\section{Presentaci\'on general}
\normalsize{ \indent
En la actualidad, la Universidad Nacional de Misiones
gestiona los papeles de los servicios a trav\'es
de documentos generados en un programa ofim\'atico;
sin haber otro tipo de informatizaci\'on sobre
el proceso. Al no tener un sistema implantado para
este prop\'osito, se carece de control sobre las
distintas etapas en que la realizaci\'on de un
servicio se encuentra.
}
\newline
\normalsize{ \indent
Los servicios pueden clasificarse en aquellos
generados por \'ordenes y por convenio.
Las \'ordenes de servicio son documentos que
constatan la vinculaci\'on entre Comitentes y
Responsables T\'ecnicos, agregando que la persona
a cargo de la Secretar\'ia de Extensi\'on y
Vinculaci\'on Tecnol\'ogica est\'a en conocimiento
de esto.
}
\newline
\normalsize{ \indent
Adicionalmente, el tiempo que transcurre ente cada
uno de los pasos a seguir en el procedimiento de un
servicio es amplio; considerando las formas en que
las solicitudes de servicio llegan, sea por los
docentes que obtienen las solicitudes de servicios
o por el acercamiento presencial del solicitante.
}
\newline
\normalsize{ \indent
Para el producto software a desarrollar, se va
a considerar:
}
\begin{itemize}
	\item Seguimiento de los servicios por
	convenio.
	\item Generaci\'on de la orden de servicio.
	\item Registro de los costos involucrados en
	el servicio.
	\item Negociaci\'on de los recursos,
	compromisos y retribuciones de un servicio.
	\item Firma de las \'ordenes de servicio.
	\item Anexo de factura y recibo
	correspondiente por orden de servicio.
\end{itemize}
\ \newline
\normalsize{ \indent
Las cuestiones que no ser\'an abarcadas en, al
menos, una versi\'on inicial son:
}
\begin{itemize}
	\item Revisi\'on de ART de los trabajadores
	de planta o con relaci\'on de dependencia,
	y seguros de monotributistas o becarios.
\end{itemize}
\ \newline
\normalsize{ \indent
Las limitaciones definitivas del sistema (es
decir, que nunca ser\'an parte de este software)
son:
}
\begin{itemize}
	\item La facturaci\'on de la orden de
	servicio, que ya se realiza por un sistema
	de AFIP.
	\item Medios de cobro a trav\'es del sistema
	(tales como MercadoPago u otros medios de
	transferencia) para las \'ordenes de servicio.
	\item Medios de pago para retribuir a los
	miembros que prestaron su servicio.
\end{itemize}
\section{Participantes del proyecto}
\normalsize{ \indent
El desarrollador a cargo de este proyecto es
Rodrigo Lionel Antoniak, mientras que el cliente
es la Universidad Nacional de Misiones.
}
\section{Objetivos del sistema}
\begin{center}
\begin{tabular}{ | p{4.5cm} | p{11cm} | }
	\hline
	\rowcolor{lightgray}
	\hfil \textbf{\textit{OBJ-01}} &
	\hfil \textbf{Objetivo}  \\
	\hline
	\raggedleft \textit{Descripci\'on} &
	Disponer de una plantilla de orden
	de servicio parametrizada, para generar
	el documento de forma consistente. \\
	\hline
	\raggedleft \textit{Estabilidad} & Alta \\
	\hline
	\raggedleft \textit{Comentarios} & Ninguno \\
	\hline
\end{tabular} \\[1cm]
\begin{tabular}{ | p{4.5cm} | p{11cm} | }
	\hline
	\rowcolor{lightgray}
	\hfil \textbf{\textit{OBJ-02}} &
	\hfil \textbf{Objetivo}  \\
	\hline
	\raggedleft \textit{Descripci\'on} &
	Otorgar un v\'ia m\'as formal y
	r\'apida para acordar los t\'erminos
	en que se regir\'a un servicio. \\
	\hline
	\raggedleft \textit{Estabilidad} & Alta \\
	\hline
	\raggedleft \textit{Comentarios} & Ninguno \\
	\hline
\end{tabular} \\[1cm]
\begin{tabular}{ | p{4.5cm} | p{11cm} | }
	\hline
	\rowcolor{lightgray}
	\hfil \textbf{\textit{OBJ-03}} &
	\hfil \textbf{Objetivo}  \\
	\hline
	\raggedleft \textit{Descripci\'on} &
	Controlar la validez de las firmas
	en los documentos firmados para las
	\'ordenes de servicio. \\
	\hline
	\raggedleft \textit{Estabilidad} & Baja \\
	\hline
	\raggedleft \textit{Comentarios} & Ninguno \\
	\hline
\end{tabular} \\[1cm]
\begin{tabular}{ | p{4.5cm} | p{11cm} | }
	\hline
	\rowcolor{lightgray}
	\hfil \textbf{\textit{OBJ-04}} &
	\hfil \textbf{Objetivo}  \\
	\hline
	\raggedleft \textit{Descripci\'on} &
	Definir condiciones en donde se
	suspenda una orden de servicio o su
	solicitud. \\
	\hline
	\raggedleft \textit{Estabilidad} & Alta \\
	\hline
	\raggedleft \textit{Comentarios} & Ninguno \\
	\hline
\end{tabular} \\[1cm]
\begin{tabular}{ | p{4.5cm} | p{11cm} | }
	\hline
	\rowcolor{lightgray}
	\hfil \textbf{\textit{OBJ-05}} &
	\hfil \textbf{Objetivo}  \\
	\hline
	\raggedleft \textit{Descripci\'on} &
	Poseer los servicios realizados
	por orden y por convenios en un solo
	lugar. \\
	\hline
	\raggedleft \textit{Estabilidad} & Baja \\
	\hline
	\raggedleft \textit{Comentarios} & Ninguno \\
	\hline
\end{tabular} \\[1cm]
\begin{tabular}{ | p{4.5cm} | p{11cm} | }
	\hline
	\rowcolor{lightgray}
	\hfil \textbf{\textit{OBJ-06}} &
	\hfil \textbf{Objetivo}  \\
	\hline
	\raggedleft \textit{Descripci\'on} &
	Vincular cada uno de los comprobantes
	con la etapa en la que se encuentra
	un servicio. \\
	\hline
	\raggedleft \textit{Estabilidad} & Baja \\
	\hline
	\raggedleft \textit{Comentarios} & Ninguno \\
	\hline
\end{tabular}
\end{center}
\section{Subsistemas del Proyecto}
\subsection[Diagrama]{Diagrama de subsistemas}
\begin{figure}[H]
 	\begin{center}
	\begin{tikzpicture}
		\begin{umlpackage}{GesServOrConv}
			\umlemptypackage[x=0,y=0]
				{Gesti\'on de Solicitudes}
			\umlemptypackage[x=8,y=0]
				{Gesti\'on de Servicios}
			\umlemptypackage[x=0,y=-3]
				{Gesti\'on de \'Ordenes}
			\umlemptypackage[x=8,y=-3]
				{Gesti\'on de Convenios}
			\umlemptypackage[x=0,y=-6]
				{Gesti\'on de Firmas}
			\umlemptypackage[x=8,y=-6]
				{Gesti\'on de Usuarios}
			\umlemptypackage[x=4,y=-9]
				{Gesti\'on de Administraci\'on}
		\end{umlpackage}
	\end{tikzpicture}
	\end{center}
	\caption{
		Diagrama de subsistemas
	}
\end{figure}
\subsection[Descripci\'on]
	{Descripci\'on de subsistemas}
\begin{itemize}
	\item \textbf{Gesti\'on de Solicitudes:}
	Administra todas las solicitudes de servicio,
	en donde se realiza el acuerdo entre el
	Representante T\'ecnico y el Comitente sobre
	los t\'erminos a regir.
	\item \textbf{Gesti\'on de \'Ordenes:}
	Administra el proceso del servicio tras la
	aprobaci\'on de la solicitud, abarcando la
	firma de la orden de servicio.
	\item \textbf{Gesti\'on de Convenios:}
	Administra el proceso del servicio tras la
	aprobaci\'on de la solicitud, abarcando la
	resoluci\'on de los convenios.
	\item \textbf{Gesti\'on de Servicios:}
	Administra el proceso del servicio tras la
	aprobaci\'on de la solicitud, desde la
	validaci\'on de la orden de servicio hasta
	el registro de cumplimiento de los t\'erminos
	acordados.
	\item \textbf{Gesti\'on de Firmas:}
	Administra la validaci\'on de las firmas
	que se encuentran en una orden de servicio,
	tanto digitalmente como de forma manuscrita.
	\item \textbf{Gesti\'on de Usuarios:}
	Administra el acceso de cada usuario a las
	distintas p\'aginas, junto con las
	notificaciones y avisos peri\'odicos a los
	usuarios.
	\item \textbf{Gesti\'on de Administraci\'on:}
	Administra el panel de control destinado al
	administrador del sistema.
\end{itemize}
\section{Requisitos}
\subsection[Funcionales]
	{Requisitos Funcionales}
\begin{enumerate}
	\hypertarget{RF-01}{%
	\item Los Comitentes podr\'an solicitar un
	servicio para que realice la universidad.}
	\hypertarget{RF-02}{%
	\item Los Comitentes podr\'an relacionar a
	otros en un servicio.}
	\hypertarget{RF-03}{%
	\item Los Comitentes podr\'an decidir si
	permanecer vinculado a un servicio o no.}
	\hypertarget{RF-04}{%
	\item Los Comitentes podr\'an seleccionar
	los Representantes T\'ecnicos.}
	\hypertarget{RF-05}{%
	\item Los Representantes T\'ecnicos podr\'an
	decidir en qu\'e servicio tomar tal rol.}
	\hypertarget{RF-06}{%
	\item Los Representantes T\'ecnicos podr\'an
	asignar recursos en un servicio que a\'un no
	est\'a aprobado.}
	\hypertarget{RF-07}{%
	\item Los Representantes T\'ecnicos podr\'an
	proponer los compromisos y retribuciones
	para una solicitud de servicio.}
	\hypertarget{RF-08}{%
	\item Los Comitentes podr\'an decidir sobre
	una propuesta hecha por el Representante
	T\'ecnico de una solicitud de servicio.}
	\hypertarget{RF-09}{%
	\item Los Ayudantes de la Secretar\'ia de
	Extensi\'on podr\'an generar la orden de
	servicio ante una solicitud de la misma que
	se haya acordado.}
	\hypertarget{RF-10}{%
	\item El Secretario de Extensi\'on podr\'a
	cancelar la orden de servicio que a\'un no
	haya sido completamente firmada.}
	\hypertarget{RF-11}{%
	\item El Secretario de Extensi\'on podr\'a
	registrar la cancelaci\'on de un servicio
	por convenio.}
	\hypertarget{RF-12}{%
	\item Los firmantes de una orden de servicio
	podr\'an firmar el documento de forma
	manuscrita o digital.}
	\hypertarget{RF-13}{%
	\item Los Ayudantes de la Secretar\'ia de
	Extensi\'on podr\'an guardar una resoluci\'on
	para un servicio por convenio.}
	\hypertarget{RF-14}{%
	\item Los ayudantes de la Secretar\'ia de
	Extensi\'on podr\'an registrar los pagos del
	Comitente sobre un servicio.}
	\hypertarget{RF-15}{%
	\item Los Representantes T\'ecnicos podr\'an
	actualizar la completitud de los servicios
	en que est\'en adjudicados.}
	\hypertarget{RF-16}{%
	\item Los Ayudantes podr\'an registrar
	la cancelaci\'on de servicios que se 
	encuentren en estado de progreso dentro
	del sistema.}
	\hypertarget{RF-17}{%
	\item Los Representantes T\'ecnicos podr\'an
	registrar los haberes que correspondan a
	los miembros de la Unidad Ejecutora de un
	servicio.}
	\hypertarget{RF-18}{%
	\item Los Tesoreros podr\'an registrar los
	cobros de los miembros de la Unidad Ejecutora
	en un servicio.}
	\hypertarget{RF-19}{%
	\item El sistema deber\'a cancelar los
	tr\'amites que se hallen fuera del lapso
	establecidos para ser completados.}
	\hypertarget{RF-20}{%
	\item El sistema deber\'a gestionar
	los perfiles que se encuentren dentro del
	sistema.}
	\hypertarget{RF-21}{%
	\item El Administrador podr\'a gestionar
	copias de seguridad de la base de datos
	del sistema.}
	\hypertarget{RF-22}{%
	\item El sistema deber\'a comunicar sobre los
	distintos estados en que se encuentre a los
	distintos usuarios que involucre.}
\end{enumerate}
\subsection[No Funcionales]
	{Requisitos No Funcionales}
\paragraph{Requisito del Producto}
El sistema deber\'a auditar los distintos
movimientos que se realicen dentro del software.
\paragraph{Requisito Organizacional}
Las \'ordenes de servicio deber\'an incluir las
firmas de los Comitentes, los Representantes
T\'ecnicos y el Secretario de Extensi\'on.
\paragraph{Requisito Externo}
El sistema deber\'a hallarse en conformidad a
la Disposici\'on 2/2023 que public\'o la
Subsecretar\'ia de Tecnolig\'ias de la
Informaci\'on, dependiente de la Secretar\'ia de
Innovaci\'on P\'ublica; y la ley nacional 25326
de la norma 64790, proveniente del senado y la
C\'amara de Diputados.
\section{Casos de Uso}
\subsection[Diagrama]{Diagramas de Casos de Uso}
\begin{figure}[H]
	\begin{center}
	\begin{tikzpicture}
	\begin{umlsystem}[x=6, fill=cyan!48]
	{Subsistema de Gesti\'on de Solicitudes}
		\umlusecase[x=3.5]{%
			\hyperlink{CUN-01}
				{Registrar solicitud}
		}
		\umlusecase[x=4.5, y=-1.85, width=2.5cm]{%
			\hyperlink{CUN-02}
				{Aceptar cometimiento}
		}
		\umlusecase[x=4.5, y=-3.55, width=2.5cm]{%
			\hyperlink{CUN-03}
				{Rechazar cometimiento}
		}
		\umlusecase[x=4.5, y=-7, width=2.5cm]{%
			\hyperlink{CUN-04}
				{Elegir Responsables}
		}
		\umlusecase[x=4.75, y=-14.25]{
			\hyperlink{CUN-05}
				{Aceptar adjudicaci\'on}
		}
		\umlusecase[x=4.5, y=-15.5]{
			\hyperlink{CUN-06}
				{Rechazar adjudicaci\'on}
		}
		\umlusecase[x=4.5, y=-16.75]{
			\hyperlink{CUN-07}
				{Autoadjudicar solicitud}
		}
		\umlusecase[x=4.5, y=-5.25, width=2.8cm]{
			\hyperlink{CUN-08}
				{Rechazar Responsables}
		}
		\umlusecase[x=1, y=-8.5, width=2.8cm]{
			\hyperlink{CUN-09}
				{Aceptar Responsables}
		}
		\umlusecase[y=-17.5]{%
			\hyperlink{CUN-10}
				{Asignar regursos}
		}
		\umlusecase[x=5, y=-13]{%
			\hyperlink{CUN-11}
				{Proponer orden}
		}
		\umlusecase[x=1, y=-10.75]{%
			\hyperlink{CUN-12}
				{Aceptar solicitud}
		}
		\umlusecase[x=5, y=-10.25]{%
			\hyperlink{CUN-13}
				{Cancelar solicitud}
		}
		\umlusecase[y=-11.75]{%
			\hyperlink{CUN-14}
				{Renegociar solicitud}
		}
		\umlusecase[x=4.5, y=-18.25]{%
			\hyperlink{CUN-15}
				{Suspender solicitud}
		}
	\end{umlsystem}
	\umlactor[y=-8.5]{Comitente}
	\umlactor[y=-15.5]{Responsable T\'ecnico}
	\umlassoc{Comitente}{usecase-1}
	\umlassoc{Comitente}{usecase-2}
	\umlassoc{Comitente}{usecase-3}
	\umlassoc{Comitente}{usecase-4}
	\umlassoc{Responsable T\'ecnico}{usecase-5}
	\umlassoc{Responsable T\'ecnico}{usecase-6}
	\umlassoc{Responsable T\'ecnico}{usecase-7}
	\umlassoc{Comitente}{usecase-8}
	\umlassoc{Comitente}{usecase-9}
	\umlassoc{Responsable T\'ecnico}{usecase-10}
	\umlassoc{Responsable T\'ecnico}{usecase-11}
	\umlassoc{Comitente}{usecase-12}
	\umlassoc{Comitente}{usecase-13}
	\umlassoc{Comitente}{usecase-14}
	\end{tikzpicture}
	\end{center}
	\caption{%
		Diagrama de casos de uso para Subsistema
		de Gesti\'on de Solicitudes
	}
\end{figure}
\begin{figure}[H]
	\begin{center}
	\begin{tikzpicture}
	\begin{umlsystem}[x=5, fill=cyan!48]
	{Subsistema de Gesti\'on de \'Ordenes}
		\umlusecase[y=-2.5]{%
			\hyperlink{CUN-16}
				{Generar orden}
		}
		\umlusecase{%
			\hyperlink{CUN-17}
				{Firmar orden}
		}
		\umlusecase[y=-1.25]{
			\hyperlink{CUN-18}
				{Cancelar orden}
		}
		\umlusecase[y=-3.75]{%
			\hyperlink{CUN-19}
				{Subir orden}
		}
		\umlusecase[y=-5]{%
			\hyperlink{CUN-20}
				{Suspender orden}
		}
	\end{umlsystem}
	\umlactor[x=-1]{Comitente}
	\umlactor[x=-1, y=-3]{Responsable T\'ecnico}
	\umlactor[x=11, y=-1.25]{Secretario}
	\umlactor[x=11, y=-3.75]{Ayudante}
	\umlassoc{Ayudante}{usecase-16}
	\umlassoc{Comitente}{usecase-17}
	\umlassoc{Responsable T\'ecnico}{usecase-17}
	\umlassoc{Secretario}{usecase-17}
	\umlassoc{Secretario}{usecase-18}
	\umlassoc{Ayudante}{usecase-19}
	\end{tikzpicture}
	\end{center}
	\caption{%
		Diagrama de casos de uso para Subsistema
		de Gesti\'on de \'Ordenes
	}
\end{figure}
\begin{figure}[H]
	\begin{center}
	\begin{tikzpicture}
	\begin{umlsystem}[x=5, fill=cyan!48]
	{Subsistema de Gesti\'on de Convenios}
		\umlusecase{%
			\hyperlink{CUN-21}
				{Subir resoluci\'on}
		}
		\umlusecase[y=-2]{%
			\hyperlink{CUN-22}
				{Cancelar convenio}
		}
	\end{umlsystem}
	\umlactor{Ayudante}
	\umlactor[y=-2]{Secretario}
	\umlassoc{Ayudante}{usecase-21}
	\umlassoc{Secretario}{usecase-22}
	\end{tikzpicture}
	\end{center}
	\caption{%
		Diagrama de casos de uso para Subsistema
		de Gesti\'on de Convenios
	}
\end{figure}
\begin{figure}[H]
	\begin{center}
	\begin{tikzpicture}
	\begin{umlsystem}[x=5, fill=cyan!48]
	{Subsistema de Gesti\'on de Servicios}
		\umlusecase[y=-2.5]{%
			\hyperlink{CUN-23}
				{Registrar pago}
		}
		\umlusecase{%
			\hyperlink{CUN-24}
				{Registrar completitud}
		}
		\umlusecase[y=-1.25]{%
			\hyperlink{CUN-25}
				{Cancelar servicio}
		}
	\end{umlsystem}
	\umlactor[x=-1]{Responsable T\'ecnico}
	\umlactor[x=10, y=-1.25]{Secretario}
	\umlactor[x=-1, y=-2.5]{Ayudante}
	\umlassoc{Ayudante}{usecase-23}
	\umlassoc{Responsable T\'ecnico}{usecase-24}
	\umlassoc{Secretario}{usecase-25}
	\end{tikzpicture}
	\end{center}
	\caption{%
		Diagrama de casos de uso para Subsistema
		de Gesti\'on de Servicios
	}
\end{figure}
\begin{figure}[H]
	\begin{center}
	\begin{tikzpicture}
	\begin{umlsystem}[x=6, fill=cyan!48]
	{Subsistema de Gesti\'on de Firmas}
		\umlusecase{%
			\hyperlink{CUN-26}
				{Agregar firma}
		}
		\umlusecase[y=-1.5]{%
			\hyperlink{CUN-27}
				{Verificar orden}
		}
	\end{umlsystem}
	\end{tikzpicture}
	\end{center}
	\caption{%
		Diagrama de casos de uso para Subsistema
		de Gesti\'on de Firmas
	}
\end{figure}
\begin{figure}[H]
	\begin{center}
	\begin{tikzpicture}
	\begin{umlsystem}[x=6, fill=cyan!48]
	{Subsistema de Gesti\'on de Usuarios}
		\umlusecase[x=1, width=1.5cm]{%
			\hyperlink{CUN-28}
				{Registrar usuario}
		}
		\umlusecase[y=-2.25]{%
			\hyperlink{CUN-29}
				{Registrar Comitente}
		}
		\umlusecase[y=-3.5]{%
			\hyperlink{CUN-30}
				{Registrar Responsable}
		}
		\umlusecase[x=3.5, y=-5.35, width=2cm]{%
			\hyperlink{CUN-31}
				{Aprobar Comitente}
		}
		\umlusecase[x=-0.25, y=-5, width=2.5cm]{%
			\hyperlink{CUN-32}
				{Aprobar Responsable}
		}
		\umlusecase[y=-6.5]{%
			\hyperlink{CUN-33}
				{Registrar Secretario}
		}
		\umlusecase[y=-7.75]{%
			\hyperlink{CUN-34}
				{Registrar Ayudante}
		}
		\umlusecase[y=-9]{%
			\hyperlink{CUN-35}
				{Invalidar Comitente}
		}
		\umlusecase[y=-10.25]{%
			\hyperlink{CUN-36}
				{Invalidar Responsable}
		}
		\umlusecase[x=1.5, y=-11.5]{%
			\hyperlink{CUN-37}
				{Invalidar Ayudante}
		}
		\umlusecase[x=1, y=-12.75]{%
			\hyperlink{CUN-38}
				{Realizar notificaci\'on}
		}
		\umlusecase[x=1, y=-14]{%
			\hyperlink{CUN-39}
				{Avisar peri\'odicamente}
		}
	\end{umlsystem}
	\umlactor[x=13.5]{Secretario}
	\umlactor[x=13.5, y=-4]{Ayudante}
	\umlactor[x=13.5, y=-8]{Administrador}
	\umlactor{Comitente}
	\umlactor[y=-4]{Responsable T\'ecnico}
	\umlassoc{Administrador}{usecase-28}
	\umlassoc{Comitente}{usecase-28}
	\umlassoc{Responsable T\'ecnico}{usecase-28}
	\umlassoc{Secretario}{usecase-28}
	\umlassoc{Ayudante}{usecase-28}
	\umlassoc{Comitente}{usecase-29}
	\umlassoc{Responsable T\'ecnico}{usecase-30}
	\umlassoc{Administrador}{usecase-31}
	\umlassoc{Administrador}{usecase-32}
	\umlassoc{Administrador}{usecase-33}
	\umlassoc{Administrador}{usecase-34}
	\umlassoc{Administrador}{usecase-35}
	\umlassoc{Administrador}{usecase-36}
	\umlassoc{Administrador}{usecase-37}
	\end{tikzpicture}
	\end{center}
	\caption{%
		Diagrama de casos de uso para Subsistema
		de Gesti\'on de Usuarios
	}
\end{figure}
\begin{figure}[H]
	\begin{center}
	\begin{tikzpicture}
	\begin{umlsystem}[x=6, fill=cyan!48]
	{Subsistema de Gesti\'on de Administraci\'on}
		\umlusecase{%
			\hyperlink{CUN-40}
				{Realizar backup}
		}
		\umlusecase[y=-1.5]{%
			\hyperlink{CUN-41}
				{Recuperar backup}
		}
	\end{umlsystem}
	\umlactor[y=-0.75]{Administrador}
	\umlassoc{Administrador}{usecase-40}
	\umlassoc{Administrador}{usecase-41}
	\end{tikzpicture}
	\end{center}
	\caption{%
		Diagrama de casos de uso para Subsistema
		de Gesti\'on de Administraci\'on
	}
\end{figure}
\subsection[Actores]
	{Definici\'on de Actores}
\begin{center}
\begin{tabular}{ | p{3cm} | p{12.5cm} | }
	\hline
	\rowcolor{lightgray}
	\hfil \textbf{\textit{ACT-01}} &
	\hfil \textbf{Administrador} \\
	\hline
	\raggedleft \textit{Descripci\'on} &
	Parametriza los valores necesarios para que
	el sistema pueda funcionar con flexibilidad,
	pero sin perder las caracter\'isticas deseadas;
	se trata de las operaciones ABM y CRUD, de los
	par\'ametros ajustables, etc\'etera. \\
	\hline
	\raggedleft \textit{Comentarios} & El administrador
	no necesariamente es el superusuario. \\
	\hline
\end{tabular} \\[1cm]
\begin{tabular}{ | p{3cm} | p{12.5cm} | }
	\hline
	\rowcolor{lightgray}
	\hfil \textbf{\textit{ACT-02}} &
	\hfil \textbf{Representante T\'ecnico} \\
	\hline
	\raggedleft \textit{Descripci\'on} &
	Tiene el poder de adjudicaci\'on a los servicios
	que la universidad realice, sea elegido por
	el Comitente o se apunte al servicio en
	particular; asigna los recursos del servicio,
	y propone los Compromisos y Retribuciones
	de cada servicio que se halle a cargo. Tambi\'en
	es una de las partes que firma una orden de
	servicio. \\
	\hline
	\raggedleft \textit{Comentarios} & Ninguno \\
	\hline
\end{tabular} \\[1cm]
\begin{tabular}{ | p{3cm} | p{12.5cm} | }
	\hline
	\rowcolor{lightgray}
	\hfil \textbf{\textit{ACT-03}} &
	\hfil \textbf{Comitente} \\
	\hline
	\raggedleft \textit{Descripci\'on} &
	Solicita \'ordenes de servicio a la universidad,
	pudiendo elegir los Representantes T\'ecnicos
	(en caso de desear alguno/s en particular); as\'i
	como decide qu\'e acci\'on tomar ante la propuesta
	de Compromisos y Retribuciones de un servicio, sea
	renegociar, aceptar o rechazar la propuesta.
	Tambi\'en es una de las partes que firma una orden
	de servicio. \\
	\hline
	\raggedleft \textit{Comentarios} & Ninguno \\
	\hline
\end{tabular} \\[1cm]
\begin{tabular}{ | p{3cm} | p{12.5cm} | }
	\hline
	\rowcolor{lightgray}
	\hfil \textbf{\textit{ACT-04}} &
	\hfil \textbf{Secretario} \\
	\hline
	\raggedleft \textit{Descripci\'on} &
	Revisa cada una de las \'ordenes de servicio
	y otorga su firma final en el documento,
	o puede cancelar la orden de servicio. \\
	\hline
	\raggedleft \textit{Comentarios} & Ninguno \\
	\hline
\end{tabular} \\[1cm]
\begin{tabular}{ | p{3cm} | p{12.5cm} | }
	\hline
	\rowcolor{lightgray}
	\hfil \textbf{\textit{ACT-05}} &
	\hfil \textbf{Ayudante} \\
	\hline
	\raggedleft \textit{Descripci\'on} &
	Se encarga de subir las \'ordenes de servicio
	que hayan sido firmadas manuscritamente y las
	resoluciones de convenios, adem\'as de las
	facturas que corresponden por servicio.
	Tambi\'en se encarga de registrar las
	cancelaciones de servicios que estaban en
	progreso. \\
	\hline
	\raggedleft \textit{Comentarios} & Ninguno \\
	\hline
\end{tabular}
\end{center}
\subsection[Uso de Negocio]
	{Descripci\'on de Casos de Uso}
\begin{center}
\hypertarget{CUN-01}{%
\begin{tabular}{ | p{3cm} | p{12.5cm} | }
	\hline
	\rowcolor{lightgray}
	\hfil \textbf{\textit{CUN-01}} &
	\hfil \textbf{Registrar solicitud} \\
	\hline
	\raggedleft \textit{Actores} & Comitente \\
	\hline
	\raggedleft \textit{Prop\'osito} & Registrar la
	solicitud de un servicio. \\
	\hline
	\raggedleft \textit{Pre Condici\'on} & (ninguna) \\
	\hline
	\raggedleft \textit{Pos Condici\'on} & Una solicitud
	de servicio ha sido agregada. \\
	\hline
	\raggedleft \textit{Descripci\'on} &
	Un Comitente quiere solicitar un servicio,
	por lo que debe dirigirse a la secci\'on de
	solicitudes; as\'i puede moverse al apartado
	para crear una solicitud nueva, donde se
	adjunta el nombre para la solicitud y una
	descripci\'on de la misma; junto con los
	Comitentes asociados y categor\'ias
	correspondientes a la solicitud. \\
	\hline
\end{tabular}} \\[1cm]
\hypertarget{CUN-02}{%
\begin{tabular}{ | p{3cm} | p{12.5cm} | }
	\hline
	\rowcolor{lightgray}
	\hfil \textbf{\textit{CUN-02}} &
	\hfil \textbf{Aceptar cometimiento} \\
	\hline
	\raggedleft \textit{Actores} & Comitente \\
	\hline
	\raggedleft \textit{Prop\'osito} & Aceptar la
	relaci\'on como Comitente a una solicitud de
	servicio que no gener\'o. \\
	\hline
	\raggedleft \textit{Pre Condici\'on} & Un Comitente
	tuvo que asignar al actor como Comitente
	(\hyperlink{CUN-01}{CUN-01}). \\
	\hline
	\raggedleft \textit{Pos Condici\'on} & Hay un nuevo
	Comitente confirmado para un servicio solicitado. \\
	\hline
	\raggedleft \textit{Descripci\'on} &
	Un Comitente quiere aceptar cumplir tal rol en un servicio
	solicitado que no lo ha generado, por lo que debe
	seleccionar la solicitud de servicio; as\'i, acepta tener
	el puesto correspondido. \\
	\hline
\end{tabular}} \\[1cm]
\hypertarget{CUN-03}{%
\begin{tabular}{ | p{3cm} | p{12.5cm} | }
	\hline
	\rowcolor{lightgray}
	\hfil \textbf{\textit{CUN-03}} &
	\hfil \textbf{Rechazar cometimiento} \\
	\hline
	\raggedleft \textit{Actores} & Comitente \\
	\hline
	\raggedleft \textit{Prop\'osito} & Rechazar la
	relaci\'on como Comitente a una solicitud de
	servicio que no gener\'o. \\
	\hline
	\raggedleft \textit{Pre Condici\'on} & Un Comitente
	tuvo que asignar al actor como Comitente
	(\hyperlink{CUN-01}{CUN-01}). \\
	\hline
	\raggedleft \textit{Pos Condici\'on} & El Comitente
	queda desligado de un servicio solicitado. \\
	\hline
	\raggedleft \textit{Descripci\'on} &
	Un Comitente quiere rechazar cumplir tal rol en un servicio
	solicitado que no lo ha generado, por lo que debe
	seleccionar la solicitud de servicio; as\'i, rechaza tener
	el puesto correspondido. \\
	\hline
\end{tabular}} \\[1cm]
\hypertarget{CUN-04}{%
\begin{tabular}{ | p{3cm} | p{12.5cm} | }
	\hline
	\rowcolor{lightgray}
	\hfil \textbf{\textit{CUN-04}} &
	\hfil \textbf{Elegir Responsables} \\
	\hline
	\raggedleft \textit{Actores} & Comitente \\
	\hline
	\raggedleft \textit{Prop\'osito} & Seleccionar
	los Responsables T\'ecnicos deseados por el
	Comitente para un servicio. \\
	\hline
	\raggedleft \textit{Pre Condici\'on} & La solicitud
	de servicio debe existir para elegir los posibles
	Responsables T\'ecnicos
	(\hyperlink{CUN-01}{CUN-01}). \\
	\hline
	\raggedleft \textit{Pos Condici\'on} & Una solicitud
	tiene candidatos a ser Responsables T\'ecnicos. \\
	\hline
	\raggedleft \textit{Descripci\'on} &
	Un Comitente quiere manifestar los Responsables
	T\'ecnicos que desea para un servicio que \'el
	solicit\'o, por lo que debe seleccionar la solicitud
	que corresponde al servicio que se quiere destinar a
	los posibles Responsables T\'ecnicos; de ah\'i,
	apunta a las personas pretendidas para cubrir los
	cargos, donde es posible seleccionar a una o varias
	para el puesto. \\
	\hline
\end{tabular}} \\[1cm]
\hypertarget{CUN-05}{%
\begin{tabular}{ | p{3cm} | p{12.5cm} | }
	\hline
	\rowcolor{lightgray}
	\hfil \textbf{\textit{CUN-05}} &
	\hfil \textbf{Aceptar adjudicaci\'on} \\
	\hline
	\raggedleft \textit{Actores} & Responsable T\'ecnico \\
	\hline
	\raggedleft \textit{Prop\'osito} & Aceptar la
	adjudicaci\'on por un Comitente para ser Responsable
	T\'ecnico de un servicio solicitado. \\
	\hline
	\raggedleft \textit{Pre Condici\'on} & Un Comitente
	tuvo que asignar al actor como Responsable T\'ecnico
	(\hyperlink{CUN-04}{CUN-04}). \\
	\hline
	\raggedleft \textit{Pos Condici\'on} & Hay un nuevo
	Responsable T\'ecnico adjudicado para un servicio
	solicitado. \\
	\hline
	\raggedleft \textit{Descripci\'on} &
	Un Responsable T\'tecnico quiere aceptar la adjudicaci\'on
	de un Comitente a un servicio solicitado, por lo que debe
	seleccionar la solicitud de servicio; as\'i, acepta tener
	el rol correspondido. \\
	\hline
\end{tabular}} \\[1cm]
\hypertarget{CUN-06}{%
\begin{tabular}{ | p{3cm} | p{12.5cm} | }
	\hline
	\rowcolor{lightgray}
	\hfil \textbf{\textit{CUN-06}} &
	\hfil \textbf{Rechazar adjudicaci\'on} \\
	\hline
	\raggedleft \textit{Actores} & Responsable T\'ecnico \\
	\hline
	\raggedleft \textit{Prop\'osito} & Rechazar la
	adjudicaci\'on por un Comitente para ser Responsable
	T\'ecnico de un servicio solicitado. \\
	\hline
	\raggedleft \textit{Pre Condici\'on} & Un Comitente
	tuvo que asignar al actor como Responsable T\'ecnico
	(\hyperlink{CUN-04}{CUN-04}). \\
	\hline
	\raggedleft \textit{Pos Condici\'on} & El Responsable
	T\'ecnico queda desligado de un servicio solicitado. \\
	\hline
	\raggedleft \textit{Descripci\'on} &
	Un Responsable T\'tecnico quiere rechazar la adjudicaci\'on
	de un Comitente a un servicio solicitado, por lo que debe
	seleccionar la solicitud de servicio; as\'i, rechaza tener
	el rol dado. \\
	\hline
\end{tabular}} \\[1cm]
\hypertarget{CUN-07}{%
\begin{tabular}{ | p{3cm} | p{12.5cm} | }
	\hline
	\rowcolor{lightgray}
	\hfil \textbf{\textit{CUN-07}} &
	\hfil \textbf{Autoadjudicar solicitud} \\
	\hline
	\raggedleft \textit{Actores} & Responsable T\'ecnico \\
	\hline
	\raggedleft \textit{Prop\'osito} & Apuntarse a realizar
	un servicio que ha sido solicitado. \\
	\hline
	\raggedleft \textit{Pre Condici\'on} & La solicitud
	de servicio debe existir para adjudicarse como posible
	Responsable T\'ecnico (\hyperlink{CUN-01}{CUN-01}). \\
	\hline
	\raggedleft \textit{Pos Condici\'on} & Una solicitud
	tiene un nuevo candidato a ser Responsable T\'ecnico. \\
	\hline
	\raggedleft \textit{Descripci\'on} &
	Un Responsable T\'ecnico quiere apuntarse para realizar
	un servicio solicitado, por lo que debe seleccionar la
	solicitud que desea adjudicarse; de ah\'i, da a conocer
	su voluntad a realizar la solicitud pendiente. \\
	\hline
\end{tabular}} \\[1cm]
\hypertarget{CUN-08}{%
\begin{tabular}{ | p{3cm} | p{12.5cm} | }
	\hline
	\rowcolor{lightgray}
	\hfil \textbf{\textit{CUN-08}} &
	\hfil \textbf{Rechazar Responsables} \\
	\hline
	\raggedleft \textit{Actores} & Comitente \\
	\hline
	\raggedleft \textit{Prop\'osito} & Rechazar los
	Responsables T\'ecnicos que se encuentren adjudicados
	a un servicio solicitado. \\
	\hline
	\raggedleft \textit{Pre Condici\'on} & Debe haber
	Responsables T\'ecnicos adjudicados a un servicio
	solicitado (\hyperlink{CUN-07}{CUN-07}). \\
	\hline
	\raggedleft \textit{Pos Condici\'on} & Hay nuevas
	personas que han sido exentas del cargo de
	Responsable T\'ecnico. \\
	\hline
	\raggedleft \textit{Descripci\'on} &
	Un Comitente quiere rechazar a personas que est\'an
	adjudicadas como Responsables T\'ecnicos, por lo que
	debe seleccionar los posibles Responsables T\'ecnicos
	en el servicio solicitado; as\'i, decide exentarlos
	del cargo adjudicado. \\
	\hline
\end{tabular}} \\[1cm]
\hypertarget{CUN-09}{%
\begin{tabular}{ | p{3cm} | p{12.5cm} | }
	\hline
	\rowcolor{lightgray}
	\hfil \textbf{\textit{CUN-09}} &
	\hfil \textbf{Aceptar Responsables} \\
	\hline
	\raggedleft \textit{Actores} & Comitente \\
	\hline
	\raggedleft \textit{Prop\'osito} & Aceptar los
	Responsables T\'ecnicos que se encuentren adjudicados
	a un servicio solicitado. \\
	\hline
	\raggedleft \textit{Pre Condici\'on} & Debe haber
	Responsables T\'ecnicos adjudicados a un servicio
	solicitado (\hyperlink{CUN-07}{CUN-07}). \\
	\hline
	\raggedleft \textit{Pos Condici\'on} & Hay nuevas
	personas que han sido aceptadas al cargo de
	Responsable T\'ecnico. \\
	\hline
	\raggedleft \textit{Descripci\'on} &
	Un Comitente quiere aceptar a personas que est\'an
	adjudicadas como Responsables T\'ecnicos, por lo que
	debe seleccionar los posibles Responsables T\'ecnicos
	en el servicio solicitado; as\'i, decide aceptarlos
	del cargo adjudicado. \\
	\hline
\end{tabular}} \\[1cm]
\hypertarget{CUN-10}{%
\begin{tabular}{ | p{3cm} | p{12.5cm} | }
	\hline
	\rowcolor{lightgray}
	\hfil \textbf{\textit{CUN-10}} &
	\hfil \textbf{Asignar regursos} \\
	\hline
	\raggedleft \textit{Actores} & Responsable T\'ecnico \\
	\hline
	\raggedleft \textit{Prop\'osito} & Asignar cu\'ales son
	los posibles recursos a utilizar en un servicio futuro. \\
	\hline
	\raggedleft \textit{Pre Condici\'on} & La solicitud
	de servicio debe contar con, al menos, un Responsable
	T\'ecnico (\hyperlink{CUN-05}{CUN-05} o 
	\hyperlink{CUN-09}{CUN-09}). \\
	\hline
	\raggedleft \textit{Pos Condici\'on} & Una solicitud
	de servicio tiene los recursos necesarios registrados. \\
	\hline
	\raggedleft \textit{Descripci\'on} &
	Un Responsable T\'ecnico quiere adjuntar los recursos
	requeridos para un servicio solicitado, por lo que debe
	seleccionar la solicitud a la que tenga los permisos
	necesarios; entonces, agrega cada uno de los recursos
	(sea material, econ\'omico y/o humano) que se pretende
	involucrar. \\
	\hline
\end{tabular}} \\[1cm]
\hypertarget{CUN-11}{%
\begin{tabular}{ | p{3cm} | p{12.5cm} | }
	\hline
	\rowcolor{lightgray}
	\hfil \textbf{\textit{CUN-11}} &
	\hfil \textbf{Proponer orden} \\
	\hline
	\raggedleft \textit{Actores} & Responsable T\'ecnico \\
	\hline
	\raggedleft \textit{Prop\'osito} & Proponer los
	compromisos y retribuciones dentro de una solicitud
	de servicio. \\
	\hline
	\raggedleft \textit{Pre Condici\'on} & La solicitud
	de servicio debe contar con los recursos asignados
	(\hyperlink{CUN-10}{CUN-10}). \\
	\hline
	\raggedleft \textit{Pos Condici\'on} & Los recursos
	asignados para un posible servicio a futuro est\'an
	avalados por un Responsable T\'ecnico. \\
	\hline
	\raggedleft \textit{Descripci\'on} &
	Un Responsable T\'ecnico quiere manifestar su acuerdo
	con los compromisos y retribuciones, por lo que debe
	seleccionar la asignaci\'on de recursos a la que
	desea acordar; entonces, da a conocer su voluntad
	respecto a los t\'erminos y condiciones adjuntos a una
	solicitud de servicio. \\
	\hline
\end{tabular}} \\[1cm]
\hypertarget{CUN-12}{%
\begin{tabular}{ | p{3cm} | p{12.5cm} | }
	\hline
	\rowcolor{lightgray}
	\hfil \textbf{\textit{CUN-12}} &
	\hfil \textbf{Aceptar propuesta} \\
	\hline
	\raggedleft \textit{Actores} & Comitente \\
	\hline
	\raggedleft \textit{Prop\'osito} & Aceptar los
	compromisos y retribuciones para un servicio
	solicitado. \\
	\hline
	\raggedleft \textit{Pre Condici\'on} & Los t\'erminos
	y condiciones del servicio deben haber sido
	propuestos por los Responsables T\'ecnicos
	(\hyperlink{CUN-11}{CUN-11}). \\
	\hline
	\raggedleft \textit{Pos Condici\'on} & La solicitud
	de un servicio ha sido aprobada por las partes
	involucradas. \\
	\hline
	\raggedleft \textit{Descripci\'on} &
	Un Comitente quiere manifestar su acuerdo con los
	compromisos y retribuciones propuestos, por lo que debe
	seleccionar los t\'erminos y condiciones del
	servicio solicitado a los que desea acordar; entonces,
	da a conocer su voluntad respecto a los compromisos
	y retribuciones para un servicio. \\
	\hline
\end{tabular}} \\[1cm]
\hypertarget{CUN-13}{%
\begin{tabular}{ | p{3cm} | p{12.5cm} | }
	\hline
	\rowcolor{lightgray}
	\hfil \textbf{\textit{CUN-13}} &
	\hfil \textbf{Cancelar solicitud} \\
	\hline
	\raggedleft \textit{Actores} & Comitente \\
	\hline
	\raggedleft \textit{Prop\'osito} & Cancelar un servicio
	solicitado ante disconformidad con los t\'erminos y
	condiciones. \\
	\hline
	\raggedleft \textit{Pre Condici\'on} & Los compromisos
	y retribuciones del servicio deben haber sido
	propuestos por los Responsables T\'ecnicos
	(\hyperlink{CUN-11}{CUN-11}). \\
	\hline
	\raggedleft \textit{Pos Condici\'on} & La solicitud
	de un servicio ha sido cancelada por el Comitente. \\
	\hline
	\raggedleft \textit{Descripci\'on} &
	Un Comitente quiere discontinuar la solicitud de un
	servicio por disconformidad con los t\'erminos y
	condiciones propuestos, por lo que debe seleccionar los
	compromisos y retribuciones del servicio a cancelar;
	entonces, la solicitud queda revocada. \\
	\hline
\end{tabular}} \\[1cm]
\hypertarget{CUN-14}{%
\begin{tabular}{ | p{3cm} | p{12.5cm} | }
	\hline
	\rowcolor{lightgray}
	\hfil \textbf{\textit{CUN-14}} &
	\hfil \textbf{Renegociar solicitud} \\
	\hline
	\raggedleft \textit{Actores} & Comitente \\
	\hline
	\raggedleft \textit{Prop\'osito} & No aceptar un servicio
	solicitado bajo los compromisos y retribuciones
	propuestos, remarcando los cambios necesarios para su
	aprobaci\'on a futuro. \\
	\hline
	\raggedleft \textit{Pre Condici\'on} & Los t\'erminos
	y condiciones del servicio deben haber sido
	propuestos por los Responsables T\'ecnicos
	(\hyperlink{CUN-11}{CUN-11}). \\
	\hline
	\raggedleft \textit{Pos Condici\'on} & La propuesta
	de compromisos y retribuciones es negada, anexando
	las modificaciones requeridas para alguna pr\'oxima. \\
	\hline
	\raggedleft \textit{Descripci\'on} &
	Un Comitente quiere manifestar su disconformidad con
	los t\'erminos y condiciones propuestos a un servicio
	solicitado, sugiriendo los cambios que se desean para
	los compromisos y retribuciones a proponer posteriormente;
	para ello, selecciona los t\'erminos y condiciones
	a los que no concuerda. Posteriormente, da a conocer
	su discrepancia con los compromisos y retribuciones en
	cuesti\'on; por \'ultimo, agrega las modificaciones que
	desea observar para la pr\'oxima propuesta de t\'erminos
	y condiciones. \\
	\hline
\end{tabular}} \\[1cm]
\hypertarget{CUN-15}{%
\begin{tabular}{ | p{3cm} | p{12.5cm} | }
	\hline
	\rowcolor{lightgray}
	\hfil \textbf{\textit{CUN-15}} &
	\hfil \textbf{Suspender solicitud} \\
	\hline
	\raggedleft \textit{Actores} & (ninguno) \\
	\hline
	\raggedleft \textit{Prop\'osito} & Cancelar un servicio
	por inactividad. \\
	\hline
	\raggedleft \textit{Pre Condici\'on} & Ha transcurrido
	el tiempo m\'aximo estipulado para una acci\'on en
	la adjudicaci\'on de Responsables T\'ecnicos
	(\hyperlink{CUN-04}{CUN-04}, \hyperlink{CUN-05}{CUN-05}
	\hyperlink{CUN-07}{CUN-07}, \hyperlink{CUN-08}{CUN-08},
	\hyperlink{CUN-08}{CUN-08} o \hyperlink{CUN-09}{CUN-09})
	o en el ajuste de recursos que pertenece a un servicio
	solicitado (\hyperlink{CUN-10}{CUN-10},
	\hyperlink{CUN-11}{CUN-11}, \hyperlink{CUN-12}{CUN-12},
	\hyperlink{CUN-13}{CUN-13} o \hyperlink{CUN-14}{CUN-14}). \\
	\hline
	\raggedleft \textit{Pos Condici\'on} & La solicitud
	de un servicio ha sido cancelada autom\'aticamente. \\
	\hline
	\raggedleft \textit{Descripci\'on} &
	Una solicitud de servicio se encuentra en una etapa de
	su procedimiento por un tiempo que excede el m\'aximo
	estipulado, por lo que debe cancelarse la misma; entonces,
	se modifica el estado de la solicitud y se invalida. \\
	\hline
\end{tabular}} \\[1cm]
\hypertarget{CUN-16}{%
\begin{tabular}{ | p{3cm} | p{12.5cm} | }
	\hline
	\rowcolor{lightgray}
	\hfil \textbf{\textit{CUN-16}} &
	\hfil \textbf{Generar orden} \\
	\hline
	\raggedleft \textit{Actores} & Ayudante \\
	\hline
	\raggedleft \textit{Prop\'osito} & Generar una orden
	de servicio en formato impreso. \\
	\hline
	\raggedleft \textit{Pre Condici\'on} & El Comitente y
	los Responsables T\'ecnicos acordaron los Compromisos
	y Retribuciones para un servicio solicitado
	(\hyperlink{CUN-12}{CUN-12}). \\
	\hline
	\raggedleft \textit{Pos Condici\'on} & (ninguna) \\
	\hline
	\raggedleft \textit{Descripci\'on} &
	Un Ayudante decide generar una orden de servicio impresa,
	por lo que debe dirigirse a la orden de servicio en
	cuesti\'on; entonces, obtiene la orden de servicio en
	forma impresa. \\
	\hline
\end{tabular}} \\[1cm]
\hypertarget{CUN-17}{%
\begin{tabular}{ | p{3cm} | p{12.5cm} | }
	\hline
	\rowcolor{lightgray}
	\hfil \textbf{\textit{CUN-17}} &
	\hfil \textbf{Firmar orden} \\
	\hline
	\raggedleft \textit{Actores} & Comitente,
	Responsable T\'ecnico y Secretario \\
	\hline
	\raggedleft \textit{Prop\'osito} & Firmar una orden
	de servicio digitalmente. \\
	\hline
	\raggedleft \textit{Pre Condici\'on} & El Comitente y
	los Responsables T\'ecnicos acordaron los Compromisos
	y Retribuciones para un servicio solicitado
	(\hyperlink{CUN-12}{CUN-12}). \\
	\hline
	\raggedleft \textit{Pos Condici\'on} & Una orden de
	servicio cuenta con una nueva firma digital. \\
	\hline
	\raggedleft \textit{Descripci\'on} &
	Un actor decide firmar una orden de servicio digitalmente,
	por lo que debe dirigirse a la orden de servicio en
	cuesti\'on; entonces, coloca la firma en el documento
	y se agrega al conjunto de firmas que ya tenga. \\
	\hline
\end{tabular}} \\[1cm]
\hypertarget{CUN-18}{%
\begin{tabular}{ | p{3cm} | p{12.5cm} | }
	\hline
	\rowcolor{lightgray}
	\hfil \textbf{\textit{CUN-18}} &
	\hfil \textbf{Cancelar orden} \\
	\hline
	\raggedleft \textit{Actores} & Secretario \\
	\hline
	\raggedleft \textit{Prop\'osito} & Cancelar una orden
	de servicio. \\
	\hline
	\raggedleft \textit{Pre Condici\'on} & El Comitente y
	los Responsables T\'ecnicos acordaron los Compromisos
	y Retribuciones para un servicio solicitado
	(\hyperlink{CUN-12}{CUN-12}). \\
	\hline
	\raggedleft \textit{Pos Condici\'on} & Una orden de
	servicio queda cancelada. \\
	\hline
	\raggedleft \textit{Descripci\'on} &
	El Secretario decide cancelar una orden de servicio,
	por lo que debe dirigirse a la orden de servicio en
	cuesti\'on; entonces, hace efectiva la nulidad de
	una orden de servicio. \\
	\hline
\end{tabular}} \\[1cm]
\hypertarget{CUN-19}{%
\begin{tabular}{ | p{3cm} | p{12.5cm} | }
	\hline
	\rowcolor{lightgray}
	\hfil \textbf{\textit{CUN-19}} &
	\hfil \textbf{Subir orden} \\
	\hline
	\raggedleft \textit{Actores} & Ayudante \\
	\hline
	\raggedleft \textit{Prop\'osito} & Subir una orden
	de servicio firmada de forma manuscrita. \\
	\hline
	\raggedleft \textit{Pre Condici\'on} & El Comitente,
	los Responsables T\'ecnicos y el Secretario firmaron
	una orden de servicio impresa
	(\hyperlink{CUN-16}{CUN-16}). \\
	\hline
	\raggedleft \textit{Pos Condici\'on} & Una orden de
	servicio firmada de forma manuscrita se encuentra
	en el sistema. \\
	\hline
	\raggedleft \textit{Descripci\'on} &
	Un Ayudante decide subir una orden de servicio impresa
	que se encuentra con todas las firmas requeridas,
	por lo que debe cargar el escaneo del papel en cuesti\'on;
	entonces, se verifica la validez del documento y
	se agrega al servicio que corresponda. \\
	\hline
\end{tabular}} \\[1cm]
\hypertarget{CUN-20}{%
\begin{tabular}{ | p{3cm} | p{12.5cm} | }
	\hline
	\rowcolor{lightgray}
	\hfil \textbf{\textit{CUN-20}} &
	\hfil \textbf{Suspender orden} \\
	\hline
	\raggedleft \textit{Actores} & (ninguno) \\
	\hline
	\raggedleft \textit{Prop\'osito} & Cancelar una orden de
	servicio por inactividad. \\
	\hline
	\raggedleft \textit{Pre Condici\'on} & Ha transcurrido
	el tiempo m\'aximo estipulado para una acci\'on en
	la firma de una orden de servicio
	(\hyperlink{CUN-17}{CUN-17} o
	\hyperlink{CUN-19}{CUN-19}). \\
	\hline
	\raggedleft \textit{Pos Condici\'on} & La orden de un
	servicio ha sido cancelada autom\'aticamente. \\
	\hline
	\raggedleft \textit{Descripci\'on} &
	Una orden de servicio se encuentra en una etapa de
	su procedimiento por un tiempo que excede el m\'aximo
	estipulado, por lo que debe cancelarse la misma; entonces,
	se modifica el estado de la orden y se invalida. \\
	\hline
\end{tabular}} \\[1cm]
\hypertarget{CUN-21}{%
\begin{tabular}{ | p{3cm} | p{12.5cm} | }
	\hline
	\rowcolor{lightgray}
	\hfil \textbf{\textit{CUN-21}} &
	\hfil \textbf{Subir resoluci\'on} \\
	\hline
	\raggedleft \textit{Actores} & Ayudante \\
	\hline
	\raggedleft \textit{Prop\'osito} & Registrar el
	pago de un Comitente sobre un servicio. \\
	\hline
	\raggedleft \textit{Pre Condici\'on} & El Consejo
	Directivo firm\'o una resoluci\'on sobre el convenio
	de un servicio. \\
	\hline
	\raggedleft \textit{Pos Condici\'on} & Una nueva
	resoluci\'on para un servicio se encuentra en el
	sistema. \\
	\hline
	\raggedleft \textit{Descripci\'on} &
	Un Ayudante debe subir una resoluci\'on respecto al
	convenio de un servicio, por lo que debe seleccionar
	el servicio involucrado; entonces, carga la propia
	resoluci\'on firmada y queda dentro del sistema. \\
	\hline
\end{tabular}} \\[1cm]
\hypertarget{CUN-22}{%
\begin{tabular}{ | p{3cm} | p{12.5cm} | }
	\hline
	\rowcolor{lightgray}
	\hfil \textbf{\textit{CUN-22}} &
	\hfil \textbf{Cancelar convenio} \\
	\hline
	\raggedleft \textit{Actores} & Secretario \\
	\hline
	\raggedleft \textit{Prop\'osito} & Cancelar un
	convenio para un servicio. \\
	\hline
	\raggedleft \textit{Pre Condici\'on} & El Comitente y
	los Responsables T\'ecnicos acordaron los Compromisos
	y Retribuciones para un servicio solicitado
	(\hyperlink{CUN-12}{CUN-12}). \\
	\hline
	\raggedleft \textit{Pos Condici\'on} & Un convenio de
	servicio queda cancelado. \\
	\hline
	\raggedleft \textit{Descripci\'on} &
	El Secretario debe registrar la cancelaci\'on de un
	convenio de servicio por parte del Consejo Directivo,
	por lo que debe dirigirse al convenio de servicio en
	cuesti\'on; entonces, hace efectiva la nulidad de
	un convenio de servicio. \\
	\hline
\end{tabular}} \\[1cm]
\hypertarget{CUN-23}{%
\begin{tabular}{ | p{3cm} | p{12.5cm} | }
	\hline
	\rowcolor{lightgray}
	\hfil \textbf{\textit{CUN-23}} &
	\hfil \textbf{Registrar pago} \\
	\hline
	\raggedleft \textit{Actores} & Ayudante \\
	\hline
	\raggedleft \textit{Prop\'osito} & Registrar el
	pago de un Comitente sobre un servicio. \\
	\hline
	\raggedleft \textit{Pre Condici\'on} & El Comitente,
	los Responsables T\'ecnicos y el Secretario firmaron
	una orden de servicio
	(\hyperlink{CUN-17}{CUN-17} o
	\hyperlink{CUN-19}{CUN-19}). \\
	\hline
	\raggedleft \textit{Pos Condici\'on} & Un nuevo pago
	sobre un servicio se encuentra en el sistema. \\
	\hline
	\raggedleft \textit{Descripci\'on} &
	Un Ayudante debe registrar un pago realizado por un
	Comitente respecto a un servicio, por lo que debe
	seleccionar el servicio involucrado; entonces, carga
	el monto pagado del servicio y el Comitente que
	realiz\'o el pago. \\
	\hline
\end{tabular}} \\[1cm]
\hypertarget{CUN-24}{%
\begin{tabular}{ | p{3cm} | p{12.5cm} | }
	\hline
	\rowcolor{lightgray}
	\hfil \textbf{\textit{CUN-24}} &
	\hfil \textbf{Registrar completitud} \\
	\hline
	\raggedleft \textit{Actores} & Responsable T\'ecnico \\
	\hline
	\raggedleft \textit{Prop\'osito} & Registrar la
	completitud de un servicio en progreso. \\
	\hline
	\raggedleft \textit{Pre Condici\'on} & El Comitente,
	los Responsables T\'ecnicos y el Secretario firmaron
	una orden de servicio
	(\hyperlink{CUN-17}{CUN-17} o
	\hyperlink{CUN-19}{CUN-19}). \\
	\hline
	\raggedleft \textit{Pos Condici\'on} & Un nuevo
	porcentaje de completitud se encuentra en el sistema. \\
	\hline
	\raggedleft \textit{Descripci\'on} &
	Un Responsable T\'ecnico decide registrar un progreso
	realizado en un servicio, por lo que debe
	seleccionar el servicio involucrado; entonces, carga
	el porcentaje que se ha progresado y una descripci\'on
	de los avances. \\
	\hline
\end{tabular}} \\[1cm]
\hypertarget{CUN-24}{%
\begin{tabular}{ | p{3cm} | p{12.5cm} | }
	\hline
	\rowcolor{lightgray}
	\hfil \textbf{\textit{CUN-24}} &
	\hfil \textbf{Cancelar servicio} \\
	\hline
	\raggedleft \textit{Actores} & Secretario \\
	\hline
	\raggedleft \textit{Prop\'osito} & Cancelar un
	servicio que se encontraba en progreso. \\
	\hline
	\raggedleft \textit{Pre Condici\'on} & El Comitente,
	los Responsables T\'ecnicos y el Secretario firmaron
	una orden de servicio
	(\hyperlink{CUN-17}{CUN-17} o
	\hyperlink{CUN-19}{CUN-19}). \\
	\hline
	\raggedleft \textit{Pos Condici\'on} & Hay un nuevo
	servicio cancelado en el sistema. \\
	\hline
	\raggedleft \textit{Descripci\'on} &
	El Secretario debe cancelar un servicio en progreso,
	por lo que debe seleccionar el servicio a cancelar;
	entonces, ingresa la causa de la cancelaci\'on y
	confirma su decisi\'on. \\
	\hline
\end{tabular}} \\[1cm]
\hypertarget{CUN-26}{%
\begin{tabular}{ | p{3cm} | p{12.5cm} | }
	\hline
	\rowcolor{lightgray}
	\hfil \textbf{\textit{CUN-26}} &
	\hfil \textbf{Agregar firma} \\
	\hline
	\raggedleft \textit{Actores} & (ninguno) \\
	\hline
	\raggedleft \textit{Prop\'osito} & Agregar una firma
	digital a un documento. \\
	\hline
	\raggedleft \textit{Pre Condici\'on} & (ninguna) \\
	\hline
	\raggedleft \textit{Pos Condici\'on} & (ninguna) \\
	\hline
	\raggedleft \textit{Descripci\'on} &
	El sistema tiene la tarea de firmar un documento
	digitalmente. Para ello, recibe el archivo en cuesti\'on,
	la firma misma con su certificado digital y el lugar a
	ubicar la firma; entonces, se obtiene el documento
	firmado como salida. \\
	\hline
\end{tabular}} \\[1cm]
\hypertarget{CUN-27}{%
\begin{tabular}{ | p{3cm} | p{12.5cm} | }
	\hline
	\rowcolor{lightgray}
	\hfil \textbf{\textit{CUN-27}} &
	\hfil \textbf{Verificar orden} \\
	\hline
	\raggedleft \textit{Actores} & (ninguno) \\
	\hline
	\raggedleft \textit{Prop\'osito} & Verificar una
	orden de servicio firmada de forma manuscrita. \\
	\hline
	\raggedleft \textit{Pre Condici\'on} & (ninguna) \\
	\hline
	\raggedleft \textit{Pos Condici\'on} & (ninguna) \\
	\hline
	\raggedleft \textit{Descripci\'on} &
	El sistema tiene la tarea de verificar una orden
	de servicio que est\'a firmada a mano. Para ello,
	recibe el escaneo del documento en cuesti\'on;
	entonces, se encuentra la correspondencia al
	servicio registrado en el sistema, se valida la
	equivalencia de contenido de texto entre el
	documento generado y el escaneado, y se comprueba
	la existencia de firmas en los bloques que
	corresponde. \\
	\hline
\end{tabular}} \\[1cm]
\hypertarget{CUN-28}{%
\begin{tabular}{ | p{3cm} | p{12.5cm} | }
	\hline
	\rowcolor{lightgray}
	\hfil \textbf{\textit{CUN-28}} &
	\hfil \textbf{Registrar Usuario} \\
	\hline
	\raggedleft \textit{Actores} & Administrador,
	Secretario, Ayudante, Representante T\'ecnico
	y Comitente \\
	\hline
	\raggedleft \textit{Prop\'osito} & Registrar un
	usuario dentro del sistema. \\
	\hline
	\raggedleft \textit{Pre Condici\'on} & (ninguna) \\
	\hline
	\raggedleft \textit{Pos Condici\'on} & Un usuario
	se ha registrado. \\
	\hline
	\raggedleft \textit{Descripci\'on} &
	El usuario quiere registrarse como tal dentro
	del sistema. Para ello, se dirige a la p\'agina
	de inicio de sistema; de ah\'i, ingresa los datos
	necesarios de un usuario y se genera el nuevo
	usuario. \\
	\hline
\end{tabular}} \\[1cm]
\hypertarget{CUN-29}{%
\begin{tabular}{ | p{3cm} | p{12.5cm} | }
	\hline
	\rowcolor{lightgray}
	\hfil \textbf{\textit{CUN-29}} &
	\hfil \textbf{Registrar Comitente} \\
	\hline
	\raggedleft \textit{Actores} & Comitente \\
	\hline
	\raggedleft \textit{Prop\'osito} & Obtener los
	datos necesarios para un nuevo Comitente. \\
	\hline
	\raggedleft \textit{Pre Condici\'on} & (ninguna) \\
	\hline
	\raggedleft \textit{Pos Condici\'on} & Los datos de
	un posible nuevo Comitente est\'an en el sistema. \\
	\hline
	\raggedleft \textit{Descripci\'on} &
	El usuario quiere agregar los datos necesarios para
	ser Comitente en el sistema. Para ello, se dirige a la
	p\'agina para registrarse como Comitente; de ah\'i,
	ingresa los datos necesarios de un Comitente y se
	guardan los datos. \\
	\hline
\end{tabular}} \\[1cm]
\hypertarget{CUN-30}{%
\begin{tabular}{ | p{3cm} | p{12.5cm} | }
	\hline
	\rowcolor{lightgray}
	\hfil \textbf{\textit{CUN-30}} &
	\hfil \textbf{Registrar Responsable} \\
	\hline
	\raggedleft \textit{Actores} & Responsable
	T\'ecnico \\
	\hline
	\raggedleft \textit{Prop\'osito} & Obtener los
	datos necesarios para un nuevo Responsable
	T\'ecnico. \\
	\hline
	\raggedleft \textit{Pre Condici\'on} & (ninguna) \\
	\hline
	\raggedleft \textit{Pos Condici\'on} & Los datos de
	un posible nuevo Responsable T\'ecnico est\'an en el
	sistema. \\
	\hline
	\raggedleft \textit{Descripci\'on} &
	El usuario quiere agregar los datos necesarios para
	ser Responsable T\'ecnico en el sistema. Para ello,
	se dirige a la p\'agina para registrarse como
	Responsable T\'ecnico; de ah\'i, ingresa los datos
	necesarios de un Comitente y se guardan los datos. \\
	\hline
\end{tabular}} \\[1cm]
\hypertarget{CUN-31}{%
\begin{tabular}{ | p{3cm} | p{12.5cm} | }
	\hline
	\rowcolor{lightgray}
	\hfil \textbf{\textit{CUN-31}} &
	\hfil \textbf{Aprobar Comitente} \\
	\hline
	\raggedleft \textit{Actores} & Administrador \\
	\hline
	\raggedleft \textit{Prop\'osito} & Habilitar a
	un usuario acceder como Comitente. \\
	\hline
	\raggedleft \textit{Pre Condici\'on} & (ninguna) \\
	\hline
	\raggedleft \textit{Pos Condici\'on} & El Comitente
	puede acceder al sistema con tal perfil. \\
	\hline
	\raggedleft \textit{Descripci\'on} &
	El administrador debe habilitar a un usuario para
	ejercer el rol de Comitente dentro el sistema. Para
	ello, se dirige al perfil que gener\'o el usuario en
	cuesti\'on; as\'i, aprueba el uso del mismo. \\
	\hline
\end{tabular}} \\[1cm]
\hypertarget{CUN-32}{%
\begin{tabular}{ | p{3cm} | p{12.5cm} | }
	\hline
	\rowcolor{lightgray}
	\hfil \textbf{\textit{CUN-32}} &
	\hfil \textbf{Aprobar Responsable} \\
	\hline
	\raggedleft \textit{Actores} & Administrador \\
	\hline
	\raggedleft \textit{Prop\'osito} & Habilitar a
	un usuario acceder como Responsable T\'ecnico. \\
	\hline
	\raggedleft \textit{Pre Condici\'on} & (ninguna) \\
	\hline
	\raggedleft \textit{Pos Condici\'on} & El Responsable
	T\'ecnico puede acceder al sistema con tal perfil. \\
	\hline
	\raggedleft \textit{Descripci\'on} &
	El administrador debe habilitar a un usuario para
	ejercer el rol de Responsable T\'ecnico dentro el
	sistema. Para ello, se dirige al perfil que gener\'o
	el usuario en cuesti\'on; as\'i, aprueba el uso del
	mismo. \\
	\hline
\end{tabular}} \\[1cm]
\hypertarget{CUN-33}{%
\begin{tabular}{ | p{3cm} | p{12.5cm} | }
	\hline
	\rowcolor{lightgray}
	\hfil \textbf{\textit{CUN-33}} &
	\hfil \textbf{Registrar Secretario} \\
	\hline
	\raggedleft \textit{Actores} & Administrador \\
	\hline
	\raggedleft \textit{Prop\'osito} & Registrar a la
	nueva persona encargada de la Secretar\'ia de
	Extensi\'on y Vinculaci\'on Tecnol\'ogica. \\
	\hline
	\raggedleft \textit{Pre Condici\'on} & (ninguna) \\
	\hline
	\raggedleft \textit{Pos Condici\'on} & Los datos de
	un posible nuevo Comitente est\'an en el sistema. \\
	\hline
	\raggedleft \textit{Descripci\'on} &
	El administrador debe agregar los datos necesarios para
	ser Secretario en el sistema. Para ello, se dirige a la
	p\'agina para registrar al nuevo Secretario; de ah\'i,
	ingresa los datos necesarios de un Secretario y se
	deshabilita al anterior. \\
	\hline
\end{tabular}} \\[1cm]
\hypertarget{CUN-34}{%
\begin{tabular}{ | p{3cm} | p{12.5cm} | }
	\hline
	\rowcolor{lightgray}
	\hfil \textbf{\textit{CUN-34}} &
	\hfil \textbf{Registrar Ayudante} \\
	\hline
	\raggedleft \textit{Actores} & Administrador \\
	\hline
	\raggedleft \textit{Prop\'osito} & Obtener los
	datos necesarios para un nuevo Responsable
	T\'ecnico. \\
	\hline
	\raggedleft \textit{Pre Condici\'on} & (ninguna) \\
	\hline
	\raggedleft \textit{Pos Condici\'on} & Los datos de
	un posible nuevo Responsable T\'ecnico est\'an en el
	sistema. \\
	\hline
	\raggedleft \textit{Descripci\'on} &
	El usuario quiere agregar los datos necesarios para
	ser Responsable T\'ecnico en el sistema. Para ello,
	se dirige a la p\'agina para registrarse como
	Responsable T\'ecnico; de ah\'i, ingresa los datos
	necesarios de un Comitente y se guardan los datos. \\
	\hline
\end{tabular}} \\[1cm]
\hypertarget{CUN-35}{%
\begin{tabular}{ | p{3cm} | p{12.5cm} | }
	\hline
	\rowcolor{lightgray}
	\hfil \textbf{\textit{CUN-35}} &
	\hfil \textbf{Invalidar Comitente} \\
	\hline
	\raggedleft \textit{Actores} & Administrador \\
	\hline
	\raggedleft \textit{Prop\'osito} & Invalidar el
	inicio a un Comitente. \\
	\hline
	\raggedleft \textit{Pre Condici\'on} & (ninguna) \\
	\hline
	\raggedleft \textit{Pos Condici\'on} & Un Comitente
	no puede ocupar tal perfil para utilizar el sistema. \\
	\hline
	\raggedleft \textit{Descripci\'on} &
	El Administrador remover el acceso de un usuario con
	el perfil de Comitente, por lo que se dirige al
	panel de administrador; de ah\'i, selecciona el
	perfil a invalidar y se lleva a cabo tal acci\'on. \\
	\hline
\end{tabular}} \\[1cm]
\hypertarget{CUN-36}{%
\begin{tabular}{ | p{3cm} | p{12.5cm} | }
	\hline
	\rowcolor{lightgray}
	\hfil \textbf{\textit{CUN-36}} &
	\hfil \textbf{Invalidar Responsable} \\
	\hline
	\raggedleft \textit{Actores} & Administrador \\
	\hline
	\raggedleft \textit{Prop\'osito} & Invalidar el
	inicio a un Responsable T\'ecnico. \\
	\hline
	\raggedleft \textit{Pre Condici\'on} & (ninguna) \\
	\hline
	\raggedleft \textit{Pos Condici\'on} & Un Responsable
	T\'ecnici no puede ocupar tal perfil para utilizar el
	sistema. \\
	\hline
	\raggedleft \textit{Descripci\'on} &
	El Administrador remover el acceso de un usuario con
	el perfil de Responsable T\'ecnico, por lo que se dirige
	al panel de administrador; de ah\'i, selecciona el
	perfil a invalidar y se lleva a cabo tal acci\'on. \\
	\hline
\end{tabular}} \\[1cm]
\hypertarget{CUN-37}{%
\begin{tabular}{ | p{3cm} | p{12.5cm} | }
	\hline
	\rowcolor{lightgray}
	\hfil \textbf{\textit{CUN-37}} &
	\hfil \textbf{Invalidar Ayudante} \\
	\hline
	\raggedleft \textit{Actores} & Administrador \\
	\hline
	\raggedleft \textit{Prop\'osito} & Invalidar el
	inicio a un Ayudante. \\
	\hline
	\raggedleft \textit{Pre Condici\'on} & (ninguna) \\
	\hline
	\raggedleft \textit{Pos Condici\'on} & Un Ayudante
	no puede ocupar tal perfil para utilizar el sistema. \\
	\hline
	\raggedleft \textit{Descripci\'on} &
	El Administrador remover el acceso de un usuario con
	el perfil de Ayudante, por lo que se dirige al
	panel de administrador; de ah\'i, selecciona el
	perfil a invalidar y se lleva a cabo tal acci\'on. \\
	\hline
\end{tabular}} \\[1cm]
\hypertarget{CUN-38}{%
\begin{tabular}{ | p{3cm} | p{12.5cm} | }
	\hline
	\rowcolor{lightgray}
	\hfil \textbf{\textit{CUN-38}} &
	\hfil \textbf{Realizar notificaci\'on} \\
	\hline
	\raggedleft \textit{Actores} & (ninguno) \\
	\hline
	\raggedleft \textit{Prop\'osito} & Notificar a un
	usuario sobre cierto estado del sistema. \\
	\hline
	\raggedleft \textit{Pre Condici\'on} & Haber
	realizado una acci\'on en el sistema que deba
	notificarse a alg\'un usuario. \\
	\hline
	\raggedleft \textit{Pos Condici\'on} & El usuario
	tiene una nueva notificaci\'on sobre el estado
	del sistema. \\
	\hline
	\raggedleft \textit{Descripci\'on} &
	El sistema tiene que notificar a un usuario sobre
	el estado del sistema, por lo que debe recibir
	el evento que lo genera; de ah\'i, genera el
	mensaje y un enlace hacia donde tomar acciones
	ante lo ocurrido. \\
	\hline
\end{tabular}} \\[1cm]
\hypertarget{CUN-39}{%
\begin{tabular}{ | p{3cm} | p{12.5cm} | }
	\hline
	\rowcolor{lightgray}
	\hfil \textbf{\textit{CUN-39}} &
	\hfil \textbf{Avisar peri\'odicamente} \\
	\hline
	\raggedleft \textit{Actores} & (ninguno) \\
	\hline
	\raggedleft \textit{Prop\'osito} & Avisar a un
	usuario sobre cierto estado del sistema
	peri\'odicamente por correo electr\'onico. \\
	\hline
	\raggedleft \textit{Pre Condici\'on} & Haber
	realizado una acci\'on en el sistema que deba
	avisarse a alg\'un usuario. \\
	\hline
	\raggedleft \textit{Pos Condici\'on} & El usuario
	tiene un nuevo correo electr\'onico sobre el estado
	del sistema. \\
	\hline
	\raggedleft \textit{Descripci\'on} &
	El sistema tiene que avisar a un usuario sobre
	el estado del sistema por correo electr\'onico,
	por lo que debe recibir el evento que lo genera;
	de ah\'i, genera el mensaje y un enlace hacia
	donde tomar acciones ante lo ocurrido. \\
	\hline
\end{tabular}} \\[1cm]
\hypertarget{CUN-40}{%
\begin{tabular}{ | p{3cm} | p{12.5cm} | }
	\hline
	\rowcolor{lightgray}
	\hfil \textbf{\textit{CUN-40}} &
	\hfil \textbf{Realizar backup} \\
	\hline
	\raggedleft \textit{Actores} & Administrador \\
	\hline
	\raggedleft \textit{Prop\'osito} & Realizar una
	copia de la base de datos del sistema. \\
	\hline
	\raggedleft \textit{Pre Condici\'on} & (ninguna) \\
	\hline
	\raggedleft \textit{Pos Condici\'on} & Una copia de
	seguridad de la base de datos existe. \\
	\hline
	\raggedleft \textit{Descripci\'on} &
	El Administrador quiere realizar una copia de la
	base de datos del sistema, por lo que se dirige al
	panel de administrador; de ah\'i, selecciona la
	opci\'on para realizar la copia de seguridad y se
	lleva a cabo tal acci\'on. \\
	\hline
\end{tabular}} \\[1cm]
\hypertarget{CUN-41}{%
\begin{tabular}{ | p{3cm} | p{12.5cm} | }
	\hline
	\rowcolor{lightgray}
	\hfil \textbf{\textit{CUN-41}} &
	\hfil \textbf{Recuperar backup} \\
	\hline
	\raggedleft \textit{Actores} & Administrador \\
	\hline
	\raggedleft \textit{Prop\'osito} & Recuperar la
	base de datos del sistema a partir de una copia de
	seguridad. \\
	\hline
	\raggedleft \textit{Pre Condici\'on} & Haber
	realizado una copia de seguridad de la base de datos
	del sistema (\hyperlink{CUN-40}{CUN-40}). \\
	\hline
	\raggedleft \textit{Pos Condici\'on} & La base de
	datos ha sido restaurada a una copia de seguridad. \\
	\hline
	\raggedleft \textit{Descripci\'on} &
	El Administrador quiere recuperar la base de datos
	del sistema con una copia de seguridad, por lo que
	se dirige al panel de administrador; de ah\'i,
	selecciona la opci\'on para recuperar la copia de
	seguridad y se lleva a cabo tal acci\'on con el
	archivo que se elija para ello. \\
	\hline
\end{tabular}}
\end{center}
\section{Rastreabilidad}
\normalsize{ \indent
A continuaci\'on, se mostrar\'a la relaci\'on entre
cada requisito funcional y caso de uso; donde RF
representa requisito funcional y CUN significa caso
de uso de negocio.
}
{\setlength{\tabcolsep}{1mm}
\begin{center}
\begin{tiny}
\begin{longtable}{|*{23}{c|}}
	\hline
	\diagbox[innerwidth=1.5cm]{CUN}{RF} &
	\hyperlink{RF-01}{01} & \hyperlink{RF-02}{02} &
	\hyperlink{RF-03}{03} & \hyperlink{RF-04}{04} &
	\hyperlink{RF-05}{05} & \hyperlink{RF-06}{06} &
	\hyperlink{RF-07}{07} & \hyperlink{RF-08}{08} &
	\hyperlink{RF-09}{09} & \hyperlink{RF-10}{10} &
	\hyperlink{RF-11}{11} & \hyperlink{RF-12}{12} &
	\hyperlink{RF-13}{13} & \hyperlink{RF-14}{14} &
	\hyperlink{RF-15}{15} & \hyperlink{RF-16}{16} &
	\hyperlink{RF-17}{17} & \hyperlink{RF-18}{18} &
	\hyperlink{RF-19}{19} & \hyperlink{RF-20}{20} &
	\hyperlink{RF-21}{21} & \hyperlink{RF-22}{22} \\
	\hline
	\endhead
	\caption[]{Matriz de rastreabilidad RF/CUN}
	\endfoot
	\captionlistentry{Matriz de rastreabilidad RF/CUN}
	\hyperlink{CUN-01}{01} & x & x & \ & \ & \ & \ & \ & \ &
	\ & \ & \ & \ & \ & \ & \ & \ & \ & \ & \ & \ \ & \ & \\
	\hline
	\hyperlink{CUN-02}{02} & \ & \ & x & \ & \ & \ & \ & \ &
	\ & \ & \ & \ & \ & \ & \ & \ & \ & \ & \ & \ & \ & \ \\
	\hline
	\hyperlink{CUN-03}{03} & \ & \ & x & \ & \ & \ & \ & \ &
	\ & \ & \ & \ & \ & \ & \ & \ & \ & \ & \ & \ & \ & \ \\
	\hline
	\hyperlink{CUN-04}{04} & \ & \ & \ & x & \ & \ & \ & \ &
	\ & \ & \ & \ & \ & \ & \ & \ & \ & \ & \ & \ \ & \ & \\
	\hline
	\hyperlink{CUN-05}{05} & \ & \ & \ & \ & x & \ & \ & \ &
	\ & \ & \ & \ & \ & \ & \ & \ & \ & \ & \ & \ & \ & \ \\
	\hline
	\hyperlink{CUN-06}{06} & \ & \ & \ & \ & x & \ & \ & \ &
	\ & \ & \ & \ & \ & \ & \ & \ & \ & \ & \ & \ & \ & \ \\
	\hline
	\hyperlink{CUN-07}{07} & \ & \ & \ & \ & x & \ & \ & \ &
	\ & \ & \ & \ & \ & \ & \ & \ & \ & \ & \ & \ & \ & \ \\
	\hline
	\hyperlink{CUN-08}{08} & \ & \ & \ & x & \ & \ & \ & \ &
	\ & \ & \ & \ & \ & \ & \ & \ & \ & \ & \ & \ & \ & \ \\
	\hline
	\hyperlink{CUN-09}{09} & \ & \ & \ & x & \ & \ & \ & \ &
	\ & \ & \ & \ & \ & \ & \ & \ & \ & \ & \ & \ & \ & \ \\
	\hline
	\hyperlink{CUN-10}{10} & \ & \ & \ & \ & \ & x & \ & \ &
	\ & \ & \ & \ & \ & \ & \ & \ & \ & \ & \ & \ & \ & \ \\
	\hline
	\hyperlink{CUN-11}{11} & \ & \ & \ & \ & \ & \ & x & \ &
	\ & \ & \ & \ & \ & \ & \ & \ & \ & \ & \ & \ & \ & \ \\
	\hline
	\hyperlink{CUN-12}{12} & \ & \ & \ & \ & \ & \ & \ & x &
	\ & \ & \ & \ & \ & \ & \ & \ & \ & \ & \ & \ & \ & \ \\
	\hline
	\hyperlink{CUN-13}{13} & \ & \ & \ & \ & \ & \ & \ & x &
	\ & \ & \ & \ & \ & \ & \ & \ & \ & \ & \ & \ & \ & \ \\
	\hline
	\hyperlink{CUN-14}{14} & \ & \ & \ & \ & \ & \ & \ & x &
	\ & \ & \ & \ & \ & \ & \ & \ & \ & \ & \ & \ & \ & \ \\
	\hline
	\hyperlink{CUN-15}{15} & \ & \ & \ & \ & \ & \ & \ & \ &
	\ & \ & \ & \ & \ & \ & \ & \ & \ & x & \ & \ & \ & \ \\
	\hline
	\hyperlink{CUN-16}{16} & \ & \ & \ & \ & \ & \ & \ & \ &
	x & \ & \ & \ & \ & \ & \ & \ & \ & \ & \ & \ & \ & \ \\
	\hline
	\hyperlink{CUN-17}{17} & \ & \ & \ & \ & \ & \ & \ & \ &
	\ & \ & \ & x & \ & \ & \ & \ & \ & \ & \ & \ & \ & \ \\
	\hline
	\hyperlink{CUN-18}{18} & \ & \ & \ & \ & \ & \ & \ & \ &
	\ & x & \ & \ & \ & \ & \ & \ & \ & \ & \ & \ & \ & \ \\
	\hline
	\hyperlink{CUN-19}{19} & \ & \ & \ & \ & \ & \ & \ & \ &
	\ & \ & \ & \ & x & \ & \ & \ & \ & \ & \ & \ & \ & \ \\
	\hline
	\hyperlink{CUN-20}{20} & \ & \ & \ & \ & \ & \ & \ & \ &
	\ & \ & \ & \ & \ & \ & \ & \ & \ & \ & x & \ & \ & \ \\
	\hline
	\hyperlink{CUN-21}{21} & \ & \ & \ & \ & \ & \ & \ & \ &
	\ & \ & x & \ & \ & \ & \ & \ & \ & \ & \ & \ & \ & \ \\
	\hline
	\hyperlink{CUN-22}{22} & \ & \ & \ & \ & \ & \ & \ & \ &
	\ & \ & \ & \ & x & \ & \ & \ & \ & \ & \ & \ & \ & \ \\
	\hline
	\hyperlink{CUN-23}{23} & \ & \ & \ & \ & \ & \ & \ & \ &
	\ & \ & \ & \ & \ & x & \ & \ & \ & \ & \ & \ & \ & \ \\
	\hline
	\hyperlink{CUN-24}{24} & \ & \ & \ & \ & \ & \ & \ & \ &
	\ & \ & \ & \ & \ & \ & x & \ & \ & \ & \ & \ & \ & \ \\
	\hline
	\hyperlink{CUN-25}{25} & \ & \ & \ & \ & \ & \ & \ & \ &
	\ & \ & \ & \ & \ & \ & \ & x & \ & \ & \ & \ & \ & \ \\
	\hline
	\hyperlink{CUN-26}{26} & \ & \ & \ & \ & \ & \ & \ & \ &
	\ & \ & \ & \ & x & \ & \ & \ & \ & \ & \ & \ & \ & \ \\
	\hline
	\hyperlink{CUN-27}{27} & \ & \ & \ & \ & \ & \ & \ & \ &
	\ & \ & \ & \ & x & \ & \ & \ & \ & \ & \ & \ & \ & \ \\
	\hline
	\hyperlink{CUN-28}{28} & \ & \ & \ & \ & \ & \ & \ & \ &
	\ & \ & \ & \ & \ & \ & \ & \ & \ & \ & \ & x & \ & \ \\
	\hline
	\hyperlink{CUN-29}{29} & \ & \ & \ & \ & \ & \ & \ & \ &
	\ & \ & \ & \ & \ & \ & \ & \ & \ & \ & \ & x & \ & \ \\
	\hline
	\hyperlink{CUN-30}{30} & \ & \ & \ & \ & \ & \ & \ & \ &
	\ & \ & \ & \ & \ & \ & \ & \ & \ & \ & \ & x & \ & \ \\
	\hline
	\hyperlink{CUN-38}{38} & \ & \ & \ & \ & \ & \ & \ & \ &
	\ & \ & \ & \ & \ & \ & \ & \ & \ & \ & \ & \ & \ & x \\
	\hline
	\hyperlink{CUN-39}{39} & \ & \ & \ & \ & \ & \ & \ & \ &
	\ & \ & \ & \ & \ & \ & \ & \ & \ & \ & \ & \ & \ & x \\
	\hline
	\hyperlink{CUN-40}{40} & \ & \ & \ & \ & \ & \ & \ & \ &
	\ & \ & \ & \ & \ & \ & \ & \ & \ & \ & \ & \ & x & \ \\
	\hline
	\hyperlink{CUN-41}{41} & \ & \ & \ & \ & \ & \ & \ & \ &
	\ & \ & \ & \ & \ & \ & \ & \ & \ & \ & \ & \ & x & \ \\
	\hline
\end{longtable}
\end{tiny}
\end{center}
}
\clearpage
\section{Casos de Uso Extendido}
\begin{center}
\hypertarget{CUE-01}{%
\begin{longtable}{ | p{3cm} | p{6.25cm} | p{6.25cm} | }
	\hline
	\rowcolor{lightgray}
	\hfil \textbf{\textit{CUE-01}} &
	\multicolumn{2}{ p{13cm} | }
		{\hfil \textbf{Registrar solicitud}} \\
	\hline
	\endhead
	\raggedleft \textit{Extiende} & 
	\multicolumn{2}{ p{13cm} | }
		{\hyperlink{CUN-01}{CUN-01}} \\
	\hline
	\multirow{2}{3cm}{%
		\raggedleft
		\textit{Flujo t\'ipico de Eventos}
	} &
	\hfil Acci\'on del Autor &
	\hfil Respuesta del Sistema \\
	\cline{2-3} &%
	\begin{enumerate}[wide, labelwidth=!, labelindent=0cm]
		\vspace{-0.25cm}
		\item El Comitente se dirige a la
		secci\'on de solicitudes de servicios
		\vspace{1cm}
		\addtocounter{enumi}{1}
		\item El Comitente se dirige a la
		la secci\'on para agregar una nueva
		solicitud
		\vspace{0.75cm}
		\addtocounter{enumi}{1}
		\item El Comitente coloca el nombre y
		categor\'ia de la solicitud, y confirma
		la informaci\'on dada
	\end{enumerate} &%
	\begin{enumerate}[wide, labelwidth=!, labelindent=0cm]
		\addtocounter{enumi}{1}
		\item El sistema devuelve el listado de
		solicitudes de servicios con la secci\'on
		para agregar una nueva solicitud
		\vspace{0.5cm}
		\addtocounter{enumi}{1}
		\item El sistema proporciona el formulario
		para generar una solicitud de servicio nueva
		\vspace{1cm}
		\addtocounter{enumi}{1}
		\item El sistema guarda la nueva solicitud
		de servicio dada y proporciona la confirmaci\'on
		de la operaci\'on exitosa
	\end{enumerate} \\
	\hline
	\raggedleft \textit{Curso Alternativo de Eventos} &
	\multicolumn{2}{ p{13cm} | }{%
		\vspace{-0.25cm}
		\parbox{13cm}{%
		Pasos 5 y 6: El Responsable T\'ecnico completa el
		formulario de forma inadecuada. El sistema
		indica el error cometido y permanece
		en el mismo lugar.
		\vspace{0.25cm}
		}} \\
	\hline
\end{longtable}}
\newpage
\hypertarget{CUE-02}{%
\begin{longtable}{ | p{3cm} | p{6.25cm} | p{6.25cm} | }
	\hline
	\rowcolor{lightgray}
	\hfil \textbf{\textit{CUE-02}} &
	\multicolumn{2}{ p{13cm} | }
		{\hfil \textbf{Elegir Responsables}} \\
	\hline
	\endhead
	\raggedleft \textit{Extiende} & 
	\multicolumn{2}{ p{13cm} | }
		{\hyperlink{CUN-04}{CUN-04}} \\
	\hline
	\multirow{2}{3cm}{%
		\raggedleft
		\textit{Flujo t\'ipico de Eventos}
	} &
	\hfil Acci\'on del Autor &
	\hfil Respuesta del Sistema \\
	\cline{2-3} &%
	\begin{enumerate}[wide, labelwidth=!, labelindent=0cm]
		\vspace{-0.5cm}
		\item El Comitente se dirige a la
		secci\'on de solicitudes de servicios
		\vspace{1.5cm}
		\addtocounter{enumi}{1}
		\item El Comitente se dirige a la
		la secci\'on para decidir sobre los
		Responsables T\'ecnicos
		\vspace{0.75cm}
		\addtocounter{enumi}{1}
		\item El Comitente elige los Responsables
		T\'ecnicos y confirma la informaci\'on dada
	\end{enumerate} &%
	\begin{enumerate}[wide, labelwidth=!, labelindent=0cm]
		\vspace{0.5cm}
		\addtocounter{enumi}{1}
		\item El sistema devuelve el listado de
		solicitudes de servicios con la secci\'on
		para decidir sobre los Responsables T\'ecnicos
		\vspace{0.75cm}
		\addtocounter{enumi}{1}
		\item El sistema proporciona el formulario
		para decidir sobre los Responsables T\'ecnicos
		\vspace{0.75cm}
		\addtocounter{enumi}{1}
		\item El sistema guarda los cambios en la solicitud
		de servicio dada y proporciona la confirmaci\'on
		de la operaci\'on exitosa
	\end{enumerate} \\
	\hline
	\raggedleft \textit{Curso Alternativo de Eventos} &
	\multicolumn{2}{ p{13cm} | }{%
		\vspace{-0.25cm}
		\parbox{13cm}{%
		Paso 5: El Comitente decide que la solicitud est\'e
		abierta a la autoadjudicaci\'on de Responsables
		T\'ecnicos. Al confirmar la decisi\'on, se modifica
		la solicitud a abierta y proporciona la confirmaci\'on
		de la operaci\'on exitosa
		\vspace{0.25cm}
		}} \\
	\hline
\end{longtable}}
\newpage
\hypertarget{CUE-03}{%
\begin{longtable}{ | p{3cm} | p{6.25cm} | p{6.25cm} | }
	\hline
	\rowcolor{lightgray}
	\hfil \textbf{\textit{CUE-03}} &
	\multicolumn{2}{ p{13cm} | }
		{\hfil \textbf{Generar Orden}} \\
	\hline
	\endhead
	\raggedleft \textit{Extiende} & 
	\multicolumn{2}{ p{13cm} | }
		{\hyperlink{CUN-16}{CUN-16}} \\
	\hline
	\multirow{2}{3cm}{%
		\raggedleft
		\textit{Flujo t\'ipico de Eventos}
	} &
	\hfil Acci\'on del Autor &
	\hfil Respuesta del Sistema \\
	\cline{2-3} &%
	\begin{enumerate}[wide, labelwidth=!, labelindent=0cm]
		\vspace{-0.75cm}
		\item El Ayudante se dirige a la
		secci\'on de \'ordenes de servicios
		\vspace{1cm}
		\addtocounter{enumi}{1}
		\item El Ayudante se dirige a la
		la secci\'on para generar la orden
		de servicio
	\end{enumerate} &%
	\begin{enumerate}[wide, labelwidth=!, labelindent=0cm]
		\vspace{0.5cm}
		\addtocounter{enumi}{1}
		\item El sistema devuelve el listado de
		\'ordenes de servicios con la secci\'on
		para generar documentos imprimibles
		\vspace{0.75cm}
		\addtocounter{enumi}{1}
		\item El sistema proporciona el archivo para
		imprimir la orden servicio a ser firmada de
		forma manuscrita
	\end{enumerate} \\
	\hline
	\raggedleft \textit{Curso Alternativo de Eventos} &
	\multicolumn{2}{ p{13cm} | }{%
		\vspace{-0.25cm}
		\parbox{13cm}{%
		\textit{(ninguno)}
		\vspace{0.25cm}
		}} \\
	\hline
\end{longtable}}
\newpage
\hypertarget{CUE-04}{%
\begin{longtable}{ | p{3cm} | p{6.25cm} | p{6.25cm} | }
	\hline
	\rowcolor{lightgray}
	\hfil \textbf{\textit{CUE-04}} &
	\multicolumn{2}{ p{13cm} | }
		{\hfil \textbf{Firmar Orden}} \\
	\hline
	\endhead
	\raggedleft \textit{Extiende} & 
	\multicolumn{2}{ p{13cm} | }
		{\hyperlink{CUN-17}{CUN-17}} \\
	\hline
	\multirow{2}{3cm}{%
		\raggedleft
		\textit{Flujo t\'ipico de Eventos}
	} &
	\hfil Acci\'on del Autor &
	\hfil Respuesta del Sistema \\
	\cline{2-3} &%
	\begin{enumerate}[wide, labelwidth=!, labelindent=0cm]
		\vspace{-0.75cm}
		\item El actor se dirige a la
		secci\'on de \'ordenes de servicios
		\vspace{1.5cm}
		\addtocounter{enumi}{1}
		\item El actor se dirige a la
		la secci\'on para firmar la orden
		de servicio
		\vspace{1cm}
		\addtocounter{enumi}{1}
		\item El actor carga la firma digital,
		junto con la contrase\~na de la misma,
		y confirma la informaci\'on dada
	\end{enumerate} &%
	\begin{enumerate}[wide, labelwidth=!, labelindent=0cm]
		\vspace{0.5cm}
		\addtocounter{enumi}{1}
		\item El sistema devuelve el listado de
		\'ordenes de servicios con la secci\'on
		para firmar la orden de servicio
		\vspace{0.5cm}
		\addtocounter{enumi}{1}
		\item El sistema proporciona el formulario
		para firmar digitalmente la orden de servicio
		\vspace{1.5cm}
		\addtocounter{enumi}{1}
		\item El sistema genera una nueva copia de la
		orden de servicio con la firma digital y
		proporciona la confirmaci\'on de la operaci\'on
		exitosa
	\end{enumerate} \\
	\hline
	\raggedleft \textit{Curso Alternativo de Eventos} &
	\multicolumn{2}{ p{13cm} | }{%
		\vspace{-0.25cm}
		\parbox{13cm}{%
		Paso 6: El archivo para firma digital proporcionado
		no corresponde a un certificado digital privado.
		El sistema indica el error cometido y permanece
		en el mismo lugar.
		\newline
		Paso 6: La contrase\~na proporcionada no descifra
		la firma digital proporcionada. El sistema indica
		el error cometido y permanece en el mismo lugar.
		\vspace{0.25cm}
		}} \\
	\hline
\end{longtable}}
\newpage
\hypertarget{CUE-05}{%
\begin{longtable}{ | p{3cm} | p{6.25cm} | p{6.25cm} | }
	\hline
	\rowcolor{lightgray}
	\hfil \textbf{\textit{CUE-05}} &
	\multicolumn{2}{ p{13cm} | }
		{\hfil \textbf{Subir Orden}} \\
	\hline
	\endhead
	\raggedleft \textit{Extiende} & 
	\multicolumn{2}{ p{13cm} | }
		{\hyperlink{CUN-19}{CUN-19}} \\
	\hline
	\multirow{2}{3cm}{%
		\raggedleft
		\textit{Flujo t\'ipico de Eventos}
	} &
	\hfil Acci\'on del Autor &
	\hfil Respuesta del Sistema \\
	\cline{2-3} &%
	\begin{enumerate}[wide, labelwidth=!, labelindent=0cm]
		\vspace{-0.75cm}
		\item El Ayudante se dirige a la
		secci\'on de \'ordenes de servicios
		\vspace{1.5cm}
		\addtocounter{enumi}{1}
		\item El Ayudante se dirige a la
		la secci\'on para subir la orden
		de servicio
		\vspace{1cm}
		\addtocounter{enumi}{1}
		\item El Ayudante carga cada archivo que
		representa una p\'agina de la orden de
		servicio y confirma la informaci\'on dada
	\end{enumerate} &%
	\begin{enumerate}[wide, labelwidth=!, labelindent=0cm]
		\vspace{0.5cm}
		\addtocounter{enumi}{1}
		\item El sistema devuelve el listado de
		\'ordenes de servicios con la secci\'on
		para subir la orden de servicio
		\vspace{0.5cm}
		\addtocounter{enumi}{1}
		\item El sistema proporciona el formulario
		para subir la orden de servicio
		\vspace{1.5cm}
		\addtocounter{enumi}{1}
		\item El sistema corrobora que las im\'agenes
		corresponden a la orden de servicio del
		formulario y proporciona la confirmaci\'on de
		la operaci\'on exitosa
	\end{enumerate} \\
	\hline
	\raggedleft \textit{Curso Alternativo de Eventos} &
	\multicolumn{2}{ p{13cm} | }{%
		\vspace{-0.25cm}
		\parbox{13cm}{%
		Paso 6: El texto del archivo no coincide con
		la orden original. El sistema indica el error
		cometido y permanece en el mismo lugar.
		\newline
		Paso 6: El bloque de firma no se encuentra en
		el lugar de la orden original. El sistema indica
		el error cometido y permanece en el mismo lugar.
		\newline
		Paso 6: No se encuentra una firma en el bloque
		destinado para ello. El sistema indica
		el error cometido y permanece en el mismo lugar.
		\vspace{0.25cm}
		}} \\
	\hline
\end{longtable}}
\newpage
\hypertarget{CUE-06}{%
\begin{longtable}{ | p{3cm} | p{6.25cm} | p{6.25cm} | }
	\hline
	\rowcolor{lightgray}
	\hfil \textbf{\textit{CUE-06}} &
	\multicolumn{2}{ p{13cm} | }
		{\hfil \textbf{Subir Convenio}} \\
	\hline
	\endhead
	\raggedleft \textit{Extiende} & 
	\multicolumn{2}{ p{13cm} | }
		{\hyperlink{CUN-21}{CUN-21}} \\
	\hline
	\multirow{2}{3cm}{%
		\raggedleft
		\textit{Flujo t\'ipico de Eventos}
	} &
	\hfil Acci\'on del Autor &
	\hfil Respuesta del Sistema \\
	\cline{2-3} &%
	\begin{enumerate}[wide, labelwidth=!, labelindent=0cm]
		\vspace{-0.75cm}
		\item El Secretario se dirige a la
		secci\'on de convenios
		\vspace{1.5cm}
		\addtocounter{enumi}{1}
		\item El Secretario se dirige a la
		la secci\'on para subir el convenio
		\vspace{1cm}
		\addtocounter{enumi}{1}
		\item El actor carga el archivo de la
		resoluci\'on y confirma la informaci\'on
		dada
	\end{enumerate} &%
	\begin{enumerate}[wide, labelwidth=!, labelindent=0cm]
		\vspace{0.25cm}
		\addtocounter{enumi}{1}
		\item El sistema devuelve el listado de
		\'ordenes de servicios con la secci\'on
		para agregar una nueva solicitud
		\vspace{0.5cm}
		\addtocounter{enumi}{1}
		\item El sistema proporciona el formulario
		para firmar digitalmente la orden de servicio
		\vspace{1.25cm}
		\addtocounter{enumi}{1}
		\item El sistema guarda la resoluci\'on con el
		convenio correspondiente y proporciona la
		confirmaci\'on de la operaci\'on exitosa
	\end{enumerate} \\
	\hline
	\raggedleft \textit{Curso Alternativo de Eventos} &
	\multicolumn{2}{ p{13cm} | }{%
		\vspace{-0.25cm}
		\parbox{13cm}{%
		\textit{(ninguno)}
		\vspace{0.25cm}
		}} \\
	\hline
\end{longtable}}
\newpage
\hypertarget{CUE-07}{%
\begin{longtable}{ | p{3cm} | p{6.25cm} | p{6.25cm} | }
	\hline
	\rowcolor{lightgray}
	\hfil \textbf{\textit{CUE-07}} &
	\multicolumn{2}{ p{13cm} | }
		{\hfil \textbf{Registrar pago}} \\
	\hline
	\endhead
	\raggedleft \textit{Extiende} & 
	\multicolumn{2}{ p{13cm} | }
		{\hyperlink{CUN-23}{CUN-23}} \\
	\hline
	\multirow{2}{3cm}{%
		\raggedleft
		\textit{Flujo t\'ipico de Eventos}
	} &
	\hfil Acci\'on del Autor &
	\hfil Respuesta del Sistema \\
	\cline{2-3} &%
	\begin{enumerate}[wide, labelwidth=!, labelindent=0cm]
		\vspace{-0.75cm}
		\item El Ayudante se dirige a la secci\'on
		de servicios
		\vspace{1.25cm}
		\addtocounter{enumi}{1}
		\item El Ayudante se dirige a la la secci\'on
		para registrar el pago
		\vspace{0.5cm}
		\addtocounter{enumi}{1}
		\item El Ayudante ingresa el monto pagado con
		el Comitente que pag\'o y confirma la
		informaci\'on dada
	\end{enumerate} &%
	\begin{enumerate}[wide, labelwidth=!, labelindent=0cm]
		\vspace{0.25cm}
		\addtocounter{enumi}{1}
		\item El sistema devuelve el listado de
		servicios con la secci\'on para agregar
		un nuevo pago
		\vspace{0.5cm}
		\addtocounter{enumi}{1}
		\item El sistema proporciona el formulario
		para registrar el pago
		\vspace{1.25cm}
		\addtocounter{enumi}{1}
		\item El sistema guarda la nueva completitud
		en el servicio y proporciona la confirmaci\'on
		de la operaci\'on exitosa
	\end{enumerate} \\
	\hline
	\raggedleft \textit{Curso Alternativo de Eventos} &
	\multicolumn{2}{ p{13cm} | }{%
		\vspace{-0.25cm}
		\parbox{13cm}{%
		\textit{(ninguno)}
		\vspace{0.25cm}
		}} \\
	\hline
\end{longtable}}
\newpage
\hypertarget{CUE-08}{%
\begin{longtable}{ | p{3cm} | p{6.25cm} | p{6.25cm} | }
	\hline
	\rowcolor{lightgray}
	\hfil \textbf{\textit{CUE-08}} &
	\multicolumn{2}{ p{13cm} | }
		{\hfil \textbf{Registrar completitud}} \\
	\hline
	\endhead
	\raggedleft \textit{Extiende} & 
	\multicolumn{2}{ p{13cm} | }
		{\hyperlink{CUN-24}{CUN-24}} \\
	\hline
	\multirow{2}{3cm}{%
		\raggedleft
		\textit{Flujo t\'ipico de Eventos}
	} &
	\hfil Acci\'on del Autor &
	\hfil Respuesta del Sistema \\
	\cline{2-3} &%
	\begin{enumerate}[wide, labelwidth=!, labelindent=0cm]
		\vspace{-0.75cm}
		\item El Responsable T\'ecnico se
		dirige a la secci\'on de servicios
		\vspace{1.25cm}
		\addtocounter{enumi}{1}
		\item El Responsable T\'ecnico se dirige
		a la la secci\'on para registrar la completitud
		\vspace{1cm}
		\addtocounter{enumi}{1}
		\item El Responsable T\'ecnico ingresa el
		porcentaje de progreso en esta iteraci\'on
		con una descripci\'on de ello y confirma la
		informaci\'on dada
	\end{enumerate} &%
	\begin{enumerate}[wide, labelwidth=!, labelindent=0cm]
		\vspace{0.25cm}
		\addtocounter{enumi}{1}
		\item El sistema devuelve el listado de
		servicios con la secci\'on para agregar
		una nueva completitud
		\vspace{1cm}
		\addtocounter{enumi}{1}
		\item El sistema proporciona el formulario
		para registrar la completitud
		\vspace{2.25cm}
		\addtocounter{enumi}{1}
		\item El sistema guarda la nueva completitud
		en el servicio y proporciona la confirmaci\'on
		de la operaci\'on exitosa
	\end{enumerate} \\
	\hline
	\raggedleft \textit{Curso Alternativo de Eventos} &
	\multicolumn{2}{ p{13cm} | }{%
		\vspace{-0.25cm}
		\parbox{13cm}{%
		\textit{(ninguno)}
		\vspace{0.25cm}
		}} \\
	\hline
\end{longtable}}
% Registrar completitud, Registrar pago
\end{center}