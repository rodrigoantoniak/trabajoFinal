\section[Recolecci\'on]{Recolecci\'on de Relevamiento}
\subsection{Primera entrevista}
\normalsize{ \indent
La universidad realiza servicios a terceros, seg\'un se
soliciten. Un servicio puede variar desde el an\'alisis
de sustancias hasta la realizaci\'on de un software;
por lo que no hay una restricci\'on definida de los
li\'imites sobre lo que puede ser un servicio.
}
\newline
\normalsize{ \indent
Estos servicios son solicitados por los Comitentes, para
que la Unidad Ejecutora realice las actividades necesarias
a cumplir. Este \'ultimo es dirigido por los Responsables
T\'ecnicos, quienes pueden ser elegidos por los Comitentes
en la solicitud o autoadjudicarse ante la vacante disponible.
}
\newline
\normalsize{ \indent
Para acordar los servicios a realizar, actualmente se
acercan los Comitentes directamente o los Responsables
T\'ecnicos poseen alguna relaci\'on con quien necesita
el servicio y lo hacen saber. Para tener un documento
que registre la realizaci\'on de un servicio, se realiza
una orden de servicio; el cual, se genera con un paquete
ofim\'atico. El mismo tiene las firmas de los Comitentes,
Responsables T\'ecnicos y persona encargada de la
Secretar\'ia de Extensi\'on y Vinculaci\'on Tecnol\'ogica.
}
\newline
\normalsize{ \indent
La orden de servicio contiene el c\'odigo que lo identifica,
los compromisos de Comitentes y Responsables T\'ecnicos,
las retribuciones econ\'omicas, y las firmas de ellos con
la de la persona encargada de la Secretar\'ia de
Extensi\'on y Vinculaci\'on Tecnol\'ogica. Por ahora,
se realizan firmas de forma manuscrita; aunque hacerlo
digitalmente facilitar\'ia tal movimiento.
}
\newline
\normalsize{ \indent
Los t\'erminos y condiciones de la orden de servicio
se acuerdan entre los Comitentes y Responsables T\'ecnicos,
para lo cual se busca poder interactuar sobre ello en
un sistema informatizado; tras tener tal informaci\'on,
tambi\'en deber\'ia ser posible generar las \'ordenes
de servicio con una plantilla del documento.
}
\newline
\normalsize{ \indent
Una vez que se tenga la orden de servicio firmada, se
inicia la ejecuci\'on del servicio. Con ello, se va
generando la facturaci\'on por un sistema de AFIP y
se recibe los pagos por el servicio a trav\'es de
transferencias bancarias. Conforme va avanzando la
ejecuci\'on de los servicios, faltar\'ia registrar
formalmente los progresos sobre lo que se vaya haciendo.
}
\newline
\normalsize{ \indent
Respecto a los pagos para miembros de una Unidad Ejecutora,
la tesorer\'ia se encarga de tales movimientos; de
todas formas, no hay inter\'es actualmente sobre
involucrar tal sector en el sistema.
}
\subsection{Segunda entrevista}
\normalsize{ \indent
Para el sistema, se espera que gestione la soliciud,
la firma de \'ordenes y la ejecuci\'on de servicios.
}
\newline
\normalsize{ \indent
Para la solicitud de servicios, debe asegurarse que
todos los participantes est\'en conformes en ser parte
del mismo. Si todos concordaron en estar dentro, los
Representantes T\'ecnicos propondr\'ian los compromisos
y retribuciones econ\'omicas a los Comitentes; de ah\'i,
ellos deciden si aceptar, cancelar el servicio o
renegociar los t\'erminos y condiciones. Al renegociar,
se repite el mismo circuito que se describi\'o tras la
definici\'on de participantes.
}
\newline
\normalsize{ \indent
Al aceptar los compromisos y retribuciones econ\'omicas,
el sistema generar\'a la orden de servicio a firmar;
para lo cual, podr\'a ser firmado digitalmente (si todos
los participantes y la persona encargada de la Secretar\'ia
de Extensi\'on y Vinculaci\'on Tecnol\'ogica poseen
firma digital). Como alternativa, se podr\'a imprimir
el documento para ser firmado a mano; de ah\'i, ser\'a
necesario subir el escaneo. Con ello, habr\'a un
complemento a la validaci\'on humana; el cual, tambi\'en
evitar\'a la b\'usqueda de la orden de servicio al
que se deba vincular.
}
\newline
\normalsize{ \indent
Al tener el documento firmado en el sistema, comenzar\'a
la ejecuci\'on del servicio; desde este punto en adelante,
habr\'a necesidad de registrar el progreso del servicio.
Los avances sobre un servicio involucran desde los pagos
hasta el progreso que los Responsables T\'ecnicos reporten
de la ejecuci\'on.
}
\subsection{Tercera entrevista}
\normalsize{ \indent
No solamente se realizan servicios por orden, sino tambi\'en
servicios por convenios. Estos \'ultimos involucran
la firma de una resoluci\'on por el Consejo Directivo,
quienes no interactuar\'ian con el software.
}
\newline
\normalsize{ \indent
Al no haber control sobre la redacci\'on de resoluciones,
no se espera que el sistema valide el documento firmado;
as\'i como tampoco se busca el soporte de firma digital,
tras la ausencia del Consejo Directivo como actor sobre
el software.
}
\newline
\normalsize{ \indent
Estos convenios cuentan con un informe, el cual proporciona
la informaci\'on similar a las \'ordenes de servicio;
facilitando la reutilizaci\'on de la estructura de
los t\'erminos y condiciones. El resto del proceso de
servicio (es decir, las solicitudes y ejecuciones) son
iguales entre convenios y \'ordenes.
}
\newline
\normalsize{ \indent
Considerando la facilidad que proporciona los convenios
sobre las \'ordenes, se espera sugerir el uso del primer
m\'etodo; para lo cual, el sistema puede estar apto a
la tarea.
}
\subsection{Anexos}
\normalsize{ \indent
En los anexos, se presenta una
\hyperlink{anexoOrdenServicio}{orden de servicios modelo}.
}
\section[Informe]{Informe de Relevamiento}
\subsection{Primera entrevista}
\begin{itemize}
	\item \textbf{Servicio:} Acci\'on de valor que realiza
	la UNaM a Comitentes que lo soliciten.
	\item \textbf{Comitente:} Persona solicitante de un
	servicio.
	\item \textbf{Unidad Ejecutora:} Conjunto de personas
	que realiza un servicio en particular.
	\item \textbf{Responsable T\'ecnico:} Persona que
	pertenece a un subconjunto de una Unidad Ejecutora
	y est\'a a cargo de una determinada parte en la
	realizaci\'on de un servicio, cuya inclusi\'on en
	un servicio puede darse por su selecci\'on de
	parte del Comitente o por apuntarse a un servicio
	por su cuenta.
	\item \textbf{Secretar\'ia de Extensi\'on y
	Vinculaci\'on Tecnol\'ogica:} Sector de la jerarqu\'ia
	organizativa de la Facultad de Ciencias Exactas,
	Qu\'imicas y Naturales de la Universidad Nacional de
	Misiones; la cual administra los servicios a
	terceros, cuya persona encargada tiene la potestad
	de decisi\'on sobre la continuidad de los servicios.
	\item \textbf{Orden de servicio:} Documento firmado
	por los Comitentes, Responsables T\'ecnicos y
	persona encargada de la Secretar\'ia de Extensi\'on
	y Vinculaci\'on Tecnol\'ogica; donde se incluye
	los compromisos de Comitentes, Unidad Ejecutora, y
	retribuciones econ\'omicas.
\end{itemize}
\subsection{Segunda entrevista}
\begin{itemize}
	\item \textbf{Pasos en el ciclo de un servicio:}
	\begin{enumerate}
		\item \underline{Solicitud de servicio:}
		Primero, todas las partes deben estar conformes
		de encontrarse dentro de la solicitud; as\'i,
		un Responsable T\'ecnico propondr\'a los
		compromisos de Comitentes, Unidad Ejecutora, y
		retribuciones econ\'omicas. Lo siguiente es
		que todos los involucrados est\'en de acuerdo
		con los t\'erminos y condiciones para avanzar
		a la siguiente etapa; caso contrario, puede
		renegociarse lo estipulado o cancelar el
		servicio.
		\item \underline{Firma de orden de servicio:}
		Puede darse por medios digitales o con firma
		manuscrita. En caso de firmarse con tinta,
		debe validarse tal documento.
		\item \underline{Ejecuci\'on del servicio:}
		Con el servicio en realizaci\'on, se debe
		registrar los pagos y progresos.
	\end{enumerate}
\end{itemize}
\subsection{Tercera entrevista}
\begin{itemize}
	\item \textbf{Convenio:} Medio alternativo a
	la orden de servicio para documentar el acuerdo
	para realizar un servicio, firmado por el
	Consejo Directivo. Dada la facilidad de firma
	del documento, se prefiere el convenio frente
	a la orden de servicio.
\end{itemize}
\section[Conclusi\'on]{Conclusi\'on de Relevamiento}
\normalsize{ \indent
Para el sistema, hay que realizar los siguientes puntos:
}
\begin{itemize}
	\item Proporcionar seguimiento de un servicio para
	las distintas partes que conforma el mismo.
	\item Formalizar el medio de comunicaci\'on en donde
	se solicita los servicios.
	\item Modelar los t\'erminos y condiciones en que se
	constituye el acuerdo de un servicio.
	\item Generar la orden de servicio a partir de una
	plantilla.
	\item Proveer opci\'on de firma digital para las
	\'ordenes de servicio.
	\item Validar una orden de servicio firmada a mano
	por software.	
\end{itemize}
