\section{Servidor}
\normalsize{\indent
El servidor se encontrar\'ia en formato
de una m\'aquina virtual con una conexi\'on
de red puente con el hu\'esped. Esto
\'ultimo se debe al uso de m\'ultiples
puntos de acceso al sistema (gracias a
django-hosts), donde el acceso a p\'ublico
estar\'a disponible para todo p\'ublico y
el administrador provisto por Django se
accede solamente desde localhost.
}
\newline
\normalsize{\indent
El hu\'esped est\'a ubicado en la sede
central de la FCEQyN (en F\'elix de Azara,
por la ciudad de Posadas), contando
con el uso del sistema operativo Linux
Debian 12.
}
\newline
\normalsize{\indent
La cantidad de memoria asignada a la
m\'aquina virtual estar\'a delegada al
criterio del sector de mantenimiento de
servidores de la FCEQyN en el edificio
donde se encuentra el hu\'esped.
As\'i mismo, la facultad posee un proveedor
de correo electr\'onico propio; en
consecuencia, no hace falta contratar
alguno espec\'ifico para este software.
}
\newline
\normalsize{\indent
Este sistema no requiere el uso de
CDNs para la distribuci\'on de contenido
est\'atico o archivos varios (aunque
puede mejorar el rendimiento del software
en un futuro).
}
\newline
\normalsize{\indent
El gestor de tareas Celery se ejecutar\'ia
como un servicio de SystemD tras el
arranque de la m\'aquina virtual y
mientras se mantenga en ejecuci\'on la
misma. Por otro lado, el servidor Daphne
estar\'ia gestionado por Nginx y
Supervisor(d).
}